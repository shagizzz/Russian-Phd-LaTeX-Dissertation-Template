\small
\pgfplotsset{
    % discard if not/.style 2 args={
    %     x filter/.code={
    %         \edef\tempa{\thisrow{#1}}
    %         \edef\tempb{#2}
    %         \ifx\tempa\tempb
    %         \else
    %             \def\pgfmathresult{inf}
    %         \fi
    %     }
    % }
}

\pgfplotsset{
% 	%samples=15,
	width= 0.9\linewidth,
	height = 10cm,
	xlabel={Вертикальное эфф. напряжение $\sigma_1$, кПа},
	ylabel={Отн. верт. деформация $\epsilon_1$, д. е.},
% 	%extra y ticks={45},
	legend pos=north west,
	y dir=reverse, 
	y tick label style={
		/pgf/number format/.cd,
			fixed,
			fixed zerofill,
			precision=2,
		/tikz/.cd
	},
	ylabel near ticks,
	% ylabel shift = 5cm,
% 	% clip mode=individual,
	x tick label style={
        /pgf/number format/set thousands separator={\,},
		/pgf/number format/.cd,
			fixed,
			fixed zerofill
        /tikz/.cd
	},
    xticklabel={
        \pgfkeys{ /pgf/number format/fixed, 
            /pgf/number format/fixed zerofill, 
            /pgf/number format/precision=0} 
        \pgfmathparse{exp(\tick)}
        \pgfmathprintnumber{\pgfmathresult}
    }
}

{
\tiny

\begin{figure}[ht]
	\begin{tikzpicture}
		% \centering
		\begin{semilogxaxis}
		\addplot[mark=*, red] table [x=Sigma, y=Epsilon, col sep=semicolon] {data/GJ6805.csv};
		\end{semilogxaxis}
	\end{tikzpicture}
	\caption{Компрессионная кривая образца \texttt{GJ6805} (ИГЭ-6, суглинки тугопластичные)}
\end{figure}

\begin{figure}
\begin{tikzpicture}
	\begin{semilogxaxis}[y dir=reverse]
	\addplot[mark=*, red] table [x=Sigma, y=Epsilon, col sep=semicolon] {data/GJ6807.csv};
	\end{semilogxaxis}
\end{tikzpicture}
\caption{Компрессионная кривая образца \texttt{GJ6807} (ИГЭ-6, суглинки тугопластичные)}
\end{figure}

\begin{figure}
\begin{tikzpicture}
	\begin{semilogxaxis}[y dir=reverse]
	\addplot[mark=*, red] table [x=Sigma, y=Epsilon, col sep=semicolon] {data/GJ6838.csv};
	\end{semilogxaxis}
\end{tikzpicture}
\caption{Компрессионная кривая образца \texttt{GJ6838} (ИГЭ-6, суглинки тугопластичные)}
\end{figure}

\begin{figure}
\begin{tikzpicture}
	\begin{semilogxaxis}[y dir=reverse]
	\addplot[mark=*, red] table [x=Sigma, y=Epsilon, col sep=semicolon] {data/GJ6809.csv};
	\end{semilogxaxis}
\end{tikzpicture}
\caption{Компрессионная кривая образца \texttt{GJ6809} (ИГЭ-7, суглинки тугопластичные)}
\end{figure}

\begin{figure}

\begin{tikzpicture}
	\begin{semilogxaxis}[y dir=reverse]
	\addplot[mark=*, red] table [x=Sigma, y=Epsilon, col sep=semicolon] {data/GJ6810.csv};
	\end{semilogxaxis}
\end{tikzpicture}
\caption{Компрессионная кривая образца \texttt{GJ6810} (ИГЭ-7, суглинки тугопластичные)}
\end{figure}

\begin{figure}
\begin{tikzpicture}
	\begin{semilogxaxis}[y dir=reverse]
	\addplot[mark=*, red] table [x=Sigma, y=Epsilon, col sep=semicolon] {data/GJ6821.csv};
	\end{semilogxaxis}
\end{tikzpicture}
\caption{Компрессионная кривая образца \texttt{GJ6821} (ИГЭ-7, суглинки тугопластичные)}
\end{figure}

\begin{figure}
\begin{tikzpicture}
	\begin{semilogxaxis}[y dir=reverse]
	\addplot[mark=*, red] table [x=Sigma, y=Epsilon, col sep=semicolon] {data/GJ6898.csv};
	\end{semilogxaxis}
\end{tikzpicture}
\caption{Компрессионная кривая образца \texttt{GJ6898} (ИГЭ-7, суглинки тугопластичные)}
\end{figure}

\begin{figure}
\begin{tikzpicture}
	\begin{semilogxaxis}[y dir=reverse]
	\addplot[mark=*, red] table [x=Sigma, y=Epsilon, col sep=semicolon] {data/GJ6822.csv};
	\end{semilogxaxis}
\end{tikzpicture}
\caption{Компрессионная кривая образца \texttt{GJ6822} (ИГЭ-8а, глины полутвердые)}
\end{figure}

\begin{figure}
\begin{tikzpicture}
	\begin{semilogxaxis}[y dir=reverse]
	\addplot[mark=*, red] table [x=Sigma, y=Epsilon, col sep=semicolon] {data/GJ6884.csv};
	\end{semilogxaxis}
\end{tikzpicture}
\caption{Компрессионная кривая образца \texttt{GJ6884} (ИГЭ-8а, глины полутвердые)}
\end{figure}

\begin{figure}
\begin{tikzpicture}
	\begin{semilogxaxis}[y dir=reverse]
	\addplot[mark=*, red] table [x=Sigma, y=Epsilon, col sep=semicolon] {data/GJ6846.csv};
	\end{semilogxaxis}§
\end{tikzpicture}
\caption{Компрессионная кривая образца \texttt{GJ6846} (ИГЭ-8а, глины полутвердые)}
\end{figure}

\begin{figure}
\begin{tikzpicture}
	\begin{semilogxaxis}[y dir=reverse]
	\addplot[mark=*, red] table [x=Sigma, y=Epsilon, col sep=semicolon] {data/GJ6855.csv};
	\end{semilogxaxis}
\end{tikzpicture}
\caption{Компрессионная кривая образца \texttt{GJ6855} (ИГЭ-9, суглинки полутвердые)}
\end{figure}

\begin{figure}
\begin{tikzpicture}
	\begin{semilogxaxis}[y dir=reverse]
	\addplot[mark=*, red] table [x=Sigma, y=Epsilon, col sep=semicolon] {data/GJ6859.csv};
	\end{semilogxaxis}
\end{tikzpicture}
\caption{Компрессионная кривая образца \texttt{GJ6859} (ИГЭ-9, суглинки полутвердые)}
\end{figure}


\begin{figure}
\begin{tikzpicture}
	\begin{semilogxaxis}[y dir=reverse]
	\addplot[mark=*, red] table [x=Sigma, y=Epsilon, col sep=semicolon] {data/GJ6865.csv};
	\end{semilogxaxis}
\end{tikzpicture}
\caption{Компрессионная кривая образца \texttt{GJ6865} (ИГЭ-9, суглинки полутвердые)}
\end{figure}

	
\begin{figure}
\begin{tikzpicture}
	\begin{semilogxaxis}[y dir=reverse]
	\addplot[mark=*, red] table [x=Sigma, y=Epsilon, col sep=semicolon] {data/GJ68A3.csv};
	\end{semilogxaxis}
\end{tikzpicture}
\caption{Компрессионная кривая образца \texttt{GJ68A3} (ИГЭ-9, суглинки полутвердые)}
\end{figure}

	
\begin{figure}
\begin{tikzpicture}
	\begin{semilogxaxis}[y dir=reverse]
	\addplot[mark=*, red] table [x=Sigma, y=Epsilon, col sep=semicolon] {data/GJ68B7.csv};
	\end{semilogxaxis}
\end{tikzpicture}
\caption{Компрессионная кривая образца \texttt{GJ68B7} (ИГЭ-9, суглинки полутвердые)}
\end{figure}
}