\chapter*{Гидрогеологические условия} 

На исследуемом участке выделяются следующие водоносные горизонты:

1) Водоносный горизонт f,lgQIIms приурочен к песчаным прослоями в суглинках московского горизонта. Горизонт напорный и вскрывается всеми скважинами на глубине 2,7-3,1 (181,7-182,2 м) [2]. Пьезометрический уровень устанавливается на глубине 1,3-1,5м (183,2-183,7м) [2]. В глинистых грунтах горизонт имеет спорадическое распространение. Относительным водоупором служат ледниковые (моренные) и флювиогляциальные суглинки и глины. Питание водоносного горизонта происходит в основном за счет инфильтрации атмосферных осадков. Разгрузка осуществляется за пределами участка изысканий.

Химический состав воды характеризуется как сульфатно-хлоридно-гидрокарбонатный магниево-кальциевый пресный с минерализацией 0,9г/л, рН=7,9 [2]. Вода неагрессивная к бетону на портландцементе любых марок, слабоагрессивная к железобетонным конструкциям при периодическом смачивании. Отмечается средняя коррозионная агрессивность к свинцовым и высокая к алюминиевым оболочкам кабелей.

2) Водоносный горизонт нижнего мела (K1) приурочен к пескам пылеватым. Горизонт напорный и вскрывается на глубине 27,8м (157,2м) [2]. Уровень устанавливается на глубине 8,8м (176,2 м) [2]. Высота напора составляет 19,0м [2]. По фондовым данным региональным водоупором служат верхнеюрские глины.

Химический состав воды характеризуется как гидрокарбонатный магниево-кальциевый пресный с минерализацией 0,6 г/л, рН=7,2 [2]. Вода неагрессивная к бетону на портландцементе любых марок, слабоагрессивная к железобетонным конструкциям при периодическом смачивании. Отмечается низкая коррозионная агрессивность к свинцовым и средняя к алюминиевым оболочкам кабелей.

Гидравлическая связь между четвертичными и меловыми водоносными горизонтами почти повсеместно отсутствует. 