\chapter*{Введение}                         % Заголовок
\addcontentsline{toc}{chapter}{Введение}    % Добавляем его в оглавление

Город Москва расширяет свои границы с каждым годом по причине увеличения населения. 
Новому населению требуется жилье. Развитие гражданского строительства достигло поселения 
Сосенское, где в настоящее время возводятся многоэтажные дома и в связи с этим проводятся 
инженерно-геологические исследования площадок для будущего построения жилых комплексов.

Поскольку территория изначально не была густо заселена, поэтому требуется более детальная 
изученность инженерно-геологических условий района.

Исследования проводились проектно-изыскательной компанией ООО~<<ГеоГрадСтрой>>. 
Было определено геологическое строение района, изучены физические и физико-механические свойства грунтов основания. 
По предоставленным организацией данным была построена схематическая карта геологического строения.

В геологическом прошлом территория города Москвы и район исследования в том числе были 
покрыты ледниковыми покровами. Различают ледники двух эпох "--- московского и донского оледенения. 
Считается, что присутствие ледниковых покровов изменяет напряженно-деформированное состояние 
грунтов, они переуплотняются за счет веса вышележащего ледника, а после его отступления, 
грунты разгружаются. Эти процессы повлияли на настоящие физико-механические свойства грунтов 
и их напряженно-деформированное состояние. Для характеристики напряженно-деформированного 
состояния массива грунтов в следствие переуплотнения используются следующе параметры:
<<<<<<< HEAD

- напряжение предуплотнения $\sigma_c$ (максимальное условное эффективное вертикальное давление, 
которое грунт испытал в прошлом);

- напряжение переуплотнения $POP$ (preoverburden pressure);

- коэффициент переуплотнения $OCR$ (overconsolidation ratio).
=======
напряжение предуплотнения $\sigma_c$, 
напряжение переуплотнения $POP$,
коэффициент переуплотнения $OCR$.
>>>>>>> b4dd3d04ae459e06fa2565cdcd3aae7f2e85c420

Целью работы являются ознакомление с геологическим, геоморфологическим и 
гидрогеологическим строением территории вблизи поселения Сосенское, также 
нужно было приобрести навыки построения карты инженерно-геологических 
условий, проведения ряда определений физических и физико-механических 
свойств, ознакомление с методами определения характеристик переуплотнения.

%Задачами работы являются графиков компрессионных испытаний 
%по разным методам, определение более уместного метода анализа
%для исследуемых грунтов.