\chapter{Заключение}\label{ch:ch8}

В результате исследований выявлено, что геологическое строение 
изученного района представляет собой слоистую толщу:

-техногенные (насыпные) грунты (t IV);

-верхнечетвертичные покровные отложения (pr III);

-среднечетвертичные флювиогляциальные и озерно-ледниковые отложения московского горизонта (f,lg II ms);

-среднечетвертичные ледниковые отложения московского горизонта (g II ms);

-нижне-среднечетвертичные флювиогляциальные, ледниково-озерные, аллювиальные и озерные 
отложения донского-московского горизонта (f,lg I-II ds-ms);

-нижнечетвертичные ледниковые отложения донского горизонта (g I ds);

-отложения нижнего отдела меловой системы (K1).

Также было определено, что в районе поселения Сосенское развиты два водоносных горизонта: 
1) приуроченный к песчаным прослоям в суглинках f,lg II ms; 2) в пылеватых песках К1. 
Оба горизонта напорные, гидравлическая связь между ними почти повсеместно отсутствует.
Участок исследований относится к неопасным по причине отсутствия активных современных 
геологических процессов.

Были проведены попытки получения характеристик переуплотнения ($\sigma_c$, POP, OCR). 
Но на имеющемся стандартном оборудовании, позволяющем создавать вертикальные 
напряжения до 2500 кПа, этого сделать не удалось.

В итоге, изучая инженерно-геологические особенности территории поселения Сосенское, 
можно сделать вывод, что район благоприятен для строительства жилых комплексов 
с точки зрения инженерно-геологических условий, поэтому строительство в этом и близ 
расположенных районах развивается быстро и привлекает новых жителей и застройщиков.