%\clearpage                                  % В том числе гарантирует, что список литературы в оглавлении будет с правильным номером страницы
%%\hypersetup{ urlcolor=black }               % Ссылки делаем чёрными
%%\providecommand*{\BibDash}{}                % В стилях ugost2008 отключаем использование тире как разделителя
%\urlstyle{rm}                               % ссылки URL обычным шрифтом
%\ifdefmacro{\microtypesetup}{\microtypesetup{protrusion=false}}{} % не рекомендуется применять пакет микротипографики к автоматически генерируемому списку литературы
%\insertbibliofull                           % Подключаем Bib-базы: все статьи единым списком
%% Режим с подсписками
%%\insertbiblioexternal                      % Подключаем Bib-базы: статьи, не являющиеся статьями автора по теме диссертации
%% Для вывода выберите и расскомментируйте одно из двух
%%\insertbiblioauthor                        % Подключаем Bib-базы: работы автора единым списком 
%%\insertbiblioauthorgrouped                 % Подключаем Bib-базы: работы автора сгруппированные (ВАК, WoS, Scopus и т.д.)
%\ifdefmacro{\microtypesetup}{\microtypesetup{protrusion=true}}{}
%\urlstyle{tt}                               % возвращаем установки шрифта ссылок URL
%%\hypersetup{ urlcolor={urlcolor} }          % Восстанавливаем цвет ссылок

\chapter*{Список литературы}
\addcontentsline{toc}{chapter}{Список литературы}

1. ГОСТ 12248-2010 Методы лабораторного определения характеристик
прочности и деформируемости. --- М. : Стандартинформ. --- 78 с.

2. ГОСТ 12536-2014 Методы лабораторного определения гранулометрического
(зернового) и микроагрегатного состава. --- М. : Стандартинформ.
--- 24 с.

3. ГОСТ 25100-2011 Грунты. Классификация. --- М. : Стандартинформ.
--- 45 с.

4. ГОСТ 5180-2015 Методы лабораторного определения физических характеристик. --- М. 
: Стандартинформ. --- 24 с.

5. ГОСТ 58326-2018 Метод лабораторного определения параметров переуплотнения. --- М. 
: Стандартинформ. --- 17 с.

6. Грунтоведение // Под ред. В. Т. Трофимова - 6-е изд., перераб. и дополн.
(серия «Классический университетский учебник») / В. Т. Трофимов
[и др.]. — М., Изд-во МГУ и Наука, 2005. — 1024 с.

7. Крамаренко В., Никитенков А. О структурной прочности глинистых грунтов
территории Томской области// Известия ТГУ. --- 2014. ---
С. 1---16.

8. Лабораторный практикум по грунтоведению: Учебное пособие / Под ред. В.А. Королёва, 
В.Н. Широкова и В.В. Шаниной. – М.: "КДУ"{}, "Добросвет"{}, 2019. – 240 с.

9. Москва. Геология и город / Под редакцией В. И. Осипова и О. П. Медведева; 
РАН, Институт геоэкологии; Мосгоргеотрест. --- Москва : Московские учебники 
и Картолитография, 1997. --- 398 с.

10. СП 47.13330.2012 Инженерные изыскания для строительства. 
Основные положения. --- М.: Минрегион России, 2013. --- 109 с.

11. СП 131.13330.2012 Строительная климатология. --- М. --- Минстрой России, 2013. --- 109 с.

12. Терцаги К. Строительная механика грунта. --- «Госстройиздат». ---
1933. --- 396 с.

13. Технический отчет, Саларьево-парк // Мошкин Д.С. --- 2019. --- 118 с.

14. Труфанов А. Методы определения параметров переуплотнения грунтов и
их практическое применение в условиях Санкт-Петербурга // Инженерно-
геологические изыскания. --- 2014. --- С. 32---39.

15. ASTM D2435-90. Standard test method for one-dimensional consolidation 
properties of soil. Annual Book of ASTM, 2001. --- 15 с.

16. Wang L., Frost J. Dissipated strain energy method for determining
preconsolidation pressure // Canadian Geotechnical Journal. — 2011. —
Янв. — Т. 41. — С. 760—768.