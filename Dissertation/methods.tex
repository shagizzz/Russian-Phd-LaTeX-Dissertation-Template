\chapter*{Методика лабораторных исследований}

Исследованные грунты относятся к классу дисперсных грунтов, подклассу связных по типу 
к осадочным и к подтипу ледниковых. 
Поэтому опрделения физических и физико-механических свойств проводились по методикам, 
которые используются именно для дисперсных связных грунтов, определенных 
стандартами государства.

В процессе исследований проводилиь определния следующих свойств грунтов:

\textbf{1.Влажность грунта.}

Определение влажности производилось по методу высушивания до постоянной массы по ГОСТ 5180-2015. 
Для исследования необходимо: сушильный шкаф, лабораторный весы по ГОСТ 24104, 
металлические или стеклянные бюксы по ГОСТ 25336, шпатель по ГОСТ 10778. 
Перед испытанием необходимо отобрать пробу грунта массой 15-50 г, 
помещают в заранее высушенный, взвешенный (m) и пронумерованный бюкс и плотно закрывают крышкой. 
Если в исследуемом грунте присутствуют включения, 
то при отборе пробы на влажность нужно удалить все видимые включения.

В процессе проведения испытания пробу грунта взвешивают в закрытом бюксе. Затем открытый бюкс 
помещают в нагретый сушильный шкаф. Грунт
высушивают до постоянной массы при температуре (105±2)°С. После каждого высушивания закрытый бюкс 
охлаждают до температуры помещения и взвешивают.
Высушивание проводят до получения разности масс грунта с бюксом при двух последующих взвешиваниях 
не более 0,02 г. Если при последующем взвешивании 
наблюдается увеличение массы, то за результат принимают наименьшую массу.

Влажность грунта w, % вычисляют по формуле:

\[
   w = \frac{m_1-m_0}{m_0-m}*100
\]
,

где - масса влажного грунта с бюксом, г;
- масса высушенного грунта с бюксом, г;
- масса пустого бюкса, г.
Допускается выражать влажность грунта в долях единицы.
Результаты испытаний следует внести в журнал.

\textbf{2.Верхний предел пластичности.}

Определение верхнего предела плпстичности, то есть влажности грунта на границе текучести 
методом балансирного конуса по ГОСТ 5180-2015.
Границу текучести следует определять как влажность приготовленной из исследуемого грунта пасты, 
при которой балансирный конус погружается
под действием собственной массы за 5 с на глубину 10 мм.

Для исследования грунта необходимо: сушильный шкаф, лабораторный весы по ГОСТ 24104, 
металлические или стеклянные бюксы по ГОСТ 25336, балансирный конус Васильева с цилиндрической 
чашкой, фарфоровая по ГОСТ 9147 или металлическая чашка диаметром 7-8 см, шпатель по ГОСТ 10778,
ступка с пестиком по ГОСТ 9147, сито с отверстием 1 мм.

Образец грунта природной влажности натирают на мелкой терке с добавкой дистиллированной воды 
(вода должна соответствовать ГОСТ 6709 по показателям рН и удельной
электропроводности (УЭП)), если это требуется, удалив из него растительные
остатки крупнее 1 мм, отбирают из размельченного грунта методом
квартования по ГОСТ 8735 пробу массой около 100 г. При наличии в грунтовой
пасте включений размером более 1 мм требуется пропустить грунтовую пасту
сквозь сито с сеткой N 1.

Подготовленную грунтовую пасту тщательно перемешивают шпателем
и небольшими порциями плотно (без воздушных полостей) укладывают в
цилиндрическую чашку. Поверхность пасты заглаживают шпателем вровень с
краями чашки.Балансирный конус, смазанный тонким слоем вазелина, подводят к
поверхности грунтовой пасты так, чтобы его острие касалось пасты. Затем
плавно отпускают конус, позволяя ему погружаться в пасту под действием
собственного веса. Погружение конуса в пасту в течение 5 с на глубину 10 мм показывает,
что грунт имеет влажность, соответствующую границе текучести. По достижении границы текучести 
из пасты отбирают пробы массой 15-30 г для определения влажности.

\textbf{3.Нижний предел пластичности.}

Исследование нижнего предела пластичности, то есть влажности грунта на границе раскатывания 
производится по ГОСТ 5180-2015. Границу раскатываемости определяют как влажность пасты, 
изготовленной из грунта, которая при раскатывании  в жгуты диаметром 3 \si{\milli\meter} 
рападается на отдельные части длиной от 3 до 10 \si{\milli\meter}.

Для определния необходимо следующее оборудование: сушильный шкаф, лабораторный весы по ГОСТ 24104, 
металлические или стеклянные бюксы по ГОСТ 25336, балансирный конус Васильева с цилиндрической 
чашкой, фарфоровая по ГОСТ 9147 или металлическая чашка диаметром 7-8 \si{\centi\meter}, шпатель по ГОСТ 10778,
ступка с пестиком по ГОСТ 9147, сито с отверстием 1 \si{\milli\meter}, мелкая терка и вазелин.

Подготавливают грунт так же, как и при определении верхнего предела пластичности или используют 
подготовленную часть грунта с предыдущего определения массой 40-50 г.
Готовый грунт тщательно перемешивают и раскатывают ладонью по пластмассовой или стеклянной поверхности 
до образования грунта диаметром 3 \si{\milli\meter}. Также возможно раскатывание пальцами одной руки 
на лодони другой. Если жгутики при нужном диаметре не распадаются на части, то грунт собирают 
в комок и повторят раскатывание. После образования кусочков диаметром 3 \si{\milli\meter} и длиной 3-10
\si{\milli\meter} их собирают в бюкс и закрывают крышкой. Когда масса грунта в бюксе достигнет 10-15 г, 
определят влажность этого грунта.

\textbf{4.Плотность грунта.}

Определение проводилось по методу взвешивания грунта в воде по ГОСТ 5180-2015. Для измерения необходимо:
нож, лабораторные весы по ГОСТ 24104, нить, парафин, песчаная баня и штатив. Непосредственно перед 
определением сперва вырезают образец грунта объемом не менее 50 \si{\centi\meter^2} и придают ему сглаженную 
форму без острых углов. Образец обвяззывают нитью так, чтобы оставался свободный конец нити для 
подвязывания нити на штатив. В то же время разогревают парафин до температуры 57-60 \si{\degreeCelsius}.
В процессе определения в первую очередь образец грунта взвешивают, затем покрывают его плотной 
парафиновой оболочкой без пузырьков воздуха. После этого охлажденный образец в оболочке взвешивают. 
Парафинированный образец взвешивают в сосуде с водой.
Для этого над чашей весов устанавливают подставку для сосуда с водой так,
чтобы исключить ее касание к чаше весов. К серьге коромысла
подвешивают образец и опускают в сосуд с водой. Объем сосуда и длина
нити должны обеспечить полное погружение образца в воду. При этом
образец не должен касаться дна и стенок сосуда.
Взвешенный образец вынимают из воды, промокают
фильтровальной бумагой и взвешивают для проверки герметичности
оболочки. Если масса образца увеличилась более чем на 0,02 г по сравнению
с первоначальной, образец следует забраковать и повторить испытание с
другим образцом.

Для расчета плотности грунта  $\rho$,  используют формулу: 

\[
   \rho = \frac{m \rho_p \rho_w}{\rho_p (m_1-m_2)-\rho_w (m_1-m)}\
\]

где m - масса образца грунта до парафинирования, г;
$m_1$ - масса парафинированного образца грунта, г;
$m_2$ - результат взвешивания образца в воде - разность масс
парафинированного образца и вытесненной им воды, г; 
$\rho_p$ - плотность парафина, принимаемая равной 0,900 г/\si{\centi\meter^3};
$\rho_w$ - плотность воды при температуре испытаний,  г/\si{\centi\meter^3}.