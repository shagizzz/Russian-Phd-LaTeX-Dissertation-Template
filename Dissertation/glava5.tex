\chapter*{Современные геологические процессы}

В настоящее время на территории Москвы выделяют следующие неблагоприятные геологические процессы и явления: подтопление территории, развитие карстово-суффозионных и суффозионных процессов, образование оползней, оседания земной поверхности разного генезиса, эрозия, распространение слабых и пучинистых грунтов. В юго-западной части Москвы наиболее характерными геологическими и инженерно-геологическими процессами являются овражная эрозия, плоскостной смыв, небольшие оползни по бортам оврагов и мелких поверхностных водотоков.

При проведении инженерно-геологических изысканий на исследуемом участке, внешних проявлений карстово-суффозионных процессов в виде блюдец или воронок проседания обнаружено не было.

Исходя из полученных геологических и гидрогеологических данных, а также на основании классификации, введенной инструкцией по проектированию зданий и сооружений в районах г. Москвы с проявлением карстово-суффозионных процессов, участок предполагаемого строительства следует отнести к территории неопасной по степени опасности проявлений карстово-суффозионных процессов.

По классификации в СП 116.13330.2012 исследуемую площадку следует отнести к VI категории устойчивости по интенсивности провалообразования, где образование карстовых провалов невозможно из-за надежной защитной перекрывающей толщи водонепроницаемых пород.