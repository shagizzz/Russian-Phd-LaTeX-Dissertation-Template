\chapter{Современные геологические процессы территории поселения Сосенское}\label{ch:ch5}

В настоящее время на территории Москвы выделяют следующие неблагоприятные геологические процессы и явления: 
подтопление территории, развитие карстово-суффозионных и суффозионных процессов, образование оползней, 
оседания земной поверхности разного генезиса, эрозия, распространение слабых и пучинистых грунтов. 
В юго-западной части Москвы наиболее характерными геологическими и инженерно-геологическими процессами 
являются овражная эрозия, плоскостной смыв, небольшие оползни по бортам оврагов и мелких поверхностных водотоков.

При проведении инженерно-геологических изысканий на исследуемом участке, внешних проявлений карстово-суффозионных 
процессов в виде блюдец или воронок проседания обнаружено не было [1].

Исходя из полученных геологических и гидрогеологических данных, а также на основании классификации, 
введенной инструкцией по проектированию зданий и~сооружений в~районах г.~Москвы с проявлением 
карстово-суффозионных процессов, участок предполагаемого строительства следует отнести к территории 
неопасной по степени опасности проявлений карстово-суффозионных процессов.

По классификации в СП 116.13330.2012 исследуемую площадку следует отнести к \RomanNumeralCaps{6} категории устойчивости 
по интенсивности провалообразования, где образование карстовых провалов невозможно 
из-за надежной защитной перекрывающей толщи водонепроницаемых пород.

Оценка сложности инженерно-геологических условий. 

Площадка исследования расположена в пределах одного геоморфологического элемента, ее поверхность 
горизонтальна и не расчленена оврагами и речными сетями. 
Поэтому по геоморфологическим условиям площадка отвечает \RomanNumeralCaps{1} категории сложности. 
По геологическим условиям район исследования характеризуется не более четыремя различными по литологии слоями, 
с различием свойств по глубине, что указывает на \RomanNumeralCaps{2} 
категорию сложности по геологическим условиям в сфере взаимодействия 
зданий и сооружений с геологической средой. 
В месте проведения исследования вскрыты два водоносных напорных горизонта с немного 
различным химическим составом, поэтому по гидрогеологическим 
условиям  в сфере взаимодействия зданий и сооружений с геологической 
средой участок можно отнести ко \RomanNumeralCaps{2} категории сложности. 

Геологических и инженерно-геологических процессов 
в районе исследования не обнаружено, следовательно, по фактору 
<<Геологические и инженерно-геологические процессы, отрицательно влияющие на условия строительства и эксплуатации 
зданий и сооружений>> участок относится к \RomanNumeralCaps{1} категории сложности. 
Многолетнемерзлых и специфических грунтов в сфере взаимодействия зданий и сооружений с геологической средой
нет, это означает, что площадка по этому фактору отвечает 
\RomanNumeralCaps{1} категории сложности. 
В районе исследования наблюдаются достаточно обширные техногенные воздействия на территорию, 
также рядом с изучаемым участком расположен законсервированный 
полигон ТБО Саларьево, следовательно, по фактору 
<<Tехногенные воздействия и изменения освоенных территорий>> участок 
отвечает \RomanNumeralCaps{2} категории сложности.

Инженерно-геологические условия исследуемой территории можно отнести ко \RomanNumeralCaps{2} категории сложности по СП 47.13330.2012.