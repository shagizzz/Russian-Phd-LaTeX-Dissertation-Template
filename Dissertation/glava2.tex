\chapter{Геологическое строение территории поселения Сосенское}\label{ch:ch2}

\section{Стратиграфия отложений}\label{sec:ch2/sec1}

Геологический разрез исследуемой территории был изучен на глубину 
29,0~\si{\meter} и представлен следующими стратиграфическими подразделениями:

\begin{itemize}
    \item техногенные (насыпные) грунты (t H);
    \item верхнечетвертичные покровные отложения (pr III);
    \item среднечетвертичные флювиогляциальные и озерно-ледниковые отложения московского горизонта (f,lg II ms);
    \item среднечетвертичные ледниковые отложения московского горизонта (g II ms);
    \item нижне-среднечетвертичные флювиогляциальные, ледниково-озерные, аллювиальные и озерные отложения донского-московского горизонта (f,lg I-II ds-ms);
    \item нижнечетвертичные ледниковые отложения донского горизонта (g I ds);
    \item отложения нижнего отдела меловой системы ($K_1$).
\end{itemize}


Поверхность исследуемого участка покрыта почвенно-растительным слоем. 
Во всех скважинах встречены насыпные грунты (t H), представленные суглинком мягкопластичным, 
перекопанным, с прослоями песка, с включениями обломков бетона, битого кирпича, стекла, остатков 
древесины. 
Мощность насыпных грунтов составила 0,2--0,3~\si{\meter} (Технический отчет. Саларьево-парк, 2019). 
Абс. отметка подошвы насыпных грунтов равна 184,5-184,7м (Технический отчет. Саларьево-парк, 2019).

Верхнечетвертичные покровные отложения (pr III) залегают повсеместно под насыпными грунтами и 
представлены глинами полутвердыми.

Глины коричневые с прослоями и пятнами серых глин, полутвердые, местами до тугопластичных, 
с прослоями ожелезнений. 
Мощность глин составляет 1,6--1,7~\si{\meter} с абс. отм. подошвы 182,8-183,0~\si{\meter} 
(Технический отчет. Саларьево-парк, 2019).

Среднечетвертичные флювиогляциальные и озерно-ледниковые отложения московского горизонта 
(f,lg II ms) залегают ниже по разрезу и представлены переслаивающейся толщей супесей, суглинков 
тугопластичных и песков мелких.

Пески мелкие образуют линзу мощностью 0,2~\si{\meter} (Технический отчет. Саларьево-парк, 2019). Пески мелкие, светло-коричневые, 
коричневые, средней плотности, глинистые, с прослоями суглинка и супеси, средней степени водонасыщения. 
Мощность песков составляет 0,2~\si{\meter} (Технический отчет. Саларьево-парк, 2019).

Супеси пластичные вскрыты в подошве флювиогляциальных отложений в виде прослоя 
мощностью 0,6~\si{\meter} (Технический отчет. Саларьево-парк, 2019). 
Супеси светло-коричневые, коричневые, пластичные, слоистые, песчанистые.

Суглинки тугопластичные слагают основную толщу флювиогляциальных отложений в виде 
слоя мощностью 0,2--1,7~м (Технический отчет. Саларьево-парк, 2019). 
Суглинки коричневые, светло-коричневые, тугопластичные, пылеватые, слоистые.

Общая мощность флювиогляциальных отложений московского горизонта составляет 
1,3--1,7~\si{\meter} с абс. отм. подошвы 181,1--181,7~\si{\meter} 
(Технический отчет. Саларьево-парк, 2019).

Среднечетвертичные ледниковые отложения московского горизонта (g II ms) 
залегают под флювиогляциальными отложениями. 
Они представлены суглинками тугопластичными.

Суглинки от коричневых до красновато-коричневых, тугопластичные, местами полутвердые, 
песчанистые, с прослоями супеси и песка, с включениями дресвы и щебня преимущественно 
карбонатных пород до 10-15\%. В подошве с прослоями песка мелкого коричневого 
насыщенного водой. 
Их мощность колеблется от 0,9~\si{\meter} до 2,1~\si{\meter} с абс. отм. 
подошвы 179,6--180,3~\si{\meter} (Технический отчет. Саларьево-парк, 2019).

Нижне-среднечетвертичные флювиогляциальные, ледниково-озерные, аллювиальные и 
озерные отложения донского-московского горизонта (f,lg I-II ds-ms) залегают 
повсеместно под флювиогляциальными и ледниковыми отложениями московского горизонта. 
Они представлены суглинками тугопластичными в кровле и в подошве, 
и глинами полутвердыми в средней части.

Суглинки тугопластичные залегают преимущественно в кровле и подошве флювиогляциальных отложений.
Суглинки от светло-коричневых до зеленовато-коричнево-серых, тугопластичные, 
местами до полутвердых, песчанистые, слабо слоистые, с редкими включениями гравия и гальки. 
Мощность суглинков колеблется от 1,1 до 3,4~\si{\meter} (Технический отчет. Саларьево-парк, 2019).

Глины полутвердые преимущественно залегают в средней части толщи флювиогляциальных 
отложений донского-московского горизонта и вскрыты всеми скважинами.
Глины серые, темно-серые, с зеленоватым оттенком, полутвердые, 
в кровле 0,2~\si{\meter} тугопластичные, слоистые, пылеватые, 
с единичными включениями гравия и гальки. 
Мощность глин составила 2,8--3,8~\si{\meter} (Технический отчет. Саларьево-парк, 2019).

Общая мощность флювиогляциальных отложений донского-московского горизонта 
составляет 6,7--8,2~\si{\meter} с абсолютной отметкой подошвы 
171,6--173,4~\si{\meter} (Технический отчет. Саларьево-парк, 2019).

Ниже залегают нижнечетвертичные ледниковые отложения донского горизонта (g I ds). 
Они представлены мощной толщей суглинков полутвердых.

Суглинки коричневые, до темно-серо-коричневых, полутвердые, 
в кровле 0,1--0,3~\si{\meter} тугопластичные, песчанистые, 
с включениями дресвы и щебня преимущественно карбонатных 
пород до 15\% (Технический отчет. Саларьево-парк, 2019). 
Мощность этих суглинков составила 14,4~\si{\meter}, 
с абсолютной отметкой подошвы 157,2~\si{\meter} (Технический отчет. Саларьево-парк, 2019).

Завершают разрез на изученную глубину (29,0~\si{\meter}) отложения 
нижнего отдела меловой системы ($K_1$) (Технический отчет. Саларьево-парк, 2019). 
Они представлены песками пылеватыми. Пески пылеватые, темно-серые, 
плотные, глинистые, слюдистые, насыщенные водой. 
Максимально вскрытая мощность этих песков 
составила 1,2~\si{\meter} (Технический отчет. Саларьево-парк, 2019).
