\chapter{Геоморфологические условия территории поселения Сосенское}\label{ch:ch3}

Участок проведения исследований находится на юго-западе г.~Москвы на 
Теплостанской возвышенности, на сильно расчлененном овражно-балочной 
сетью рельефе (Москва. Геология и город, 1997). 

Теплостанская возвышенность относится к Москворецко-Окской пологоувалистой равнине, рельеф 
и геологическое строение которой сильно были определены характером развития московской 
стадии южной части ледникового покрова. 

Поверхность возвышенности имеет ступенчатый характер. Нижние ступени перекрыты флювиогляциальными 
и озерно-ледниковыми отложениями, с линзами морены и являют собой флювиогляциальную равнину. 
Высокие ступени перекрыты моренными отложениями московского и донского оледенения. 
Мощность четвертичных образований колеблется в среднем от 10 до 20 м, максимальная "--- менее 30 м. 
Ступени-холмы от Москвы-реки поднимаются к Теплому Стану, максимальная абсолютная отметка "--- 255,2 м. 
Абсолютные высоты ступеней 175--180, 190--200, 210--230 м (Москва. Геология и город, 1997). 
Все три ступени имеют довольно большие пологие 
и слабо расчлененные поверхности. 

Нижняя ступень "--- наибольшая по площади, представлена междуречьем Сетуни 
и Москвы с крутыми береговыми откосами. На ней находится 
Московский Государственный Университет им.~М.~В.~Ломоносова. 
На холмах средней ступени располагаются станции метро «Проспект Вернадского», 
«Каховская», «Варшавская» и другие. 
Участок проведения исследований находится непосредственно на высокой ступени. 

Для Теплостанской возвышенности характерна максимальная густота изначальной речной 
и овражно-балочной сети и максимальная глубина речных долин для Москвы. 
Овраги и балки здесь почти всегда имеют длинные и пологие приовражные, прибалочные и придолинные склоны, 
что свидетельствует о длительном процессе их формирования. Большинство крупных рек преимущественно текут 
на восток с некоторым отклонением к северу. Так текут важнейшие притоки р.~Москвы: 
р.~Филька, р.~Сетунь, р.~Чура, р.~Котловка и~др. 

Верховья большинства речек этого района располагаются вблизи самой высокой точки возвышенности, 
точнее – у метро Теплый Стан. Например, река Сосенка, которая протекает 
относительно близко к участку исследований, 
берет свое начало именно здесь. Также довольно близко от самой высокой точки 
Теплостанской возвышенности начинается речка Сетунь, которая тоже течет 
относительно близко к исследуемой территории.