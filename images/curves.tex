\pgfplotsset{
%samples=15,
width=0.45\linewidth,
xlabel={Диаметр частиц $d$, мм},
ylabel={Содержание частиц, \%},
%extra y ticks={45},
legend pos=north west,
ymin = 0,
ymax = 100,
}

\begin{figure}
	{\centering
	\small
	\subbottom[ИГЭ-6 (суглинки тугопластичные)]{\begin{tikzpicture}
    \begin{semilogxaxis}[]
    
    % ИГЭ-6
    \addplot[smooth, no marks, red!70!orange!80!black] table [x=Size, y=GJ6805, col sep=semicolon] {data/hydrometer-cumulative.csv};
    \addplot[smooth, no marks, red!70!orange!80!black] table [x=Size, y=GJ6807, col sep=semicolon] {data/hydrometer-cumulative.csv};
    \addplot[smooth, no marks, red!70!orange!80!black] table [x=Size, y=GJ6838, col sep=semicolon] {data/hydrometer-cumulative.csv};
    \addplot[smooth, no marks, red!70!orange!80!black] table [x=Size, y=GJ6835, col sep=semicolon] {data/hydrometer-cumulative.csv};
    
    \end{semilogxaxis}
\end{tikzpicture}}
	\hfill
	\subbottom[ИГЭ-7 (суглинки тугопластичные)]{\begin{tikzpicture}
    \begin{semilogxaxis}[]
    
    % ИГЭ-7
    \addplot[smooth, no marks, lime!40!black] table [x=Size, y=GJ6898, col sep=semicolon] {data/hydrometer-cumulative.csv};
    \addplot[smooth, no marks, lime!40!black] table [x=Size, y=GJ6874, col sep=semicolon] {data/hydrometer-cumulative.csv};
    
    \end{semilogxaxis}
\end{tikzpicture}
}
	}
	\\
	{\centering
	\small
	\subbottom[ИГЭ-8а (глины полутвердые)]{\begin{tikzpicture}
    \begin{semilogxaxis}[]
    
    % ИГЭ-8
    \addplot[smooth, no marks, green!10!lime!30!black] table [x=Size, y=GJ6822, col sep=semicolon] {data/hydrometer-cumulative.csv};
    \addplot[smooth, no marks, green!10!lime!30!black] table [x=Size, y=GJ6884, col sep=semicolon] {data/hydrometer-cumulative.csv};
    \addplot[smooth, no marks, green!10!lime!30!black] table [x=Size, y=GJ6846, col sep=semicolon] {data/hydrometer-cumulative.csv};
    \addplot[smooth, no marks, green!10!lime!30!black] table [x=Size, y=GJ6890, col sep=semicolon] {data/hydrometer-cumulative.csv};
    \addplot[smooth, no marks, green!10!lime!30!black] table [x=Size, y=GJ6888, col sep=semicolon] {data/hydrometer-cumulative.csv};
    
    
    \end{semilogxaxis}
\end{tikzpicture}}
	\hfill
	\subbottom[ИГЭ-9 (суглинки полутвердые)]{\begin{tikzpicture}
    \begin{semilogxaxis}[]
    
    % ИГЭ-9
    \addplot[smooth, no marks, red!80!orange!50!black] table [x=Size, y=GJ6865, col sep=semicolon] {data/hydrometer-cumulative.csv};
    \addplot[smooth, no marks, red!80!orange!50!black] table [x=Size, y=GJ68A3, col sep=semicolon] {data/hydrometer-cumulative.csv};
    \addplot[smooth, no marks, red!80!orange!50!black] table [x=Size, y=GJ68B7, col sep=semicolon] {data/hydrometer-cumulative.csv};
    \addplot[smooth, no marks, red!80!orange!50!black] table [x=Size, y=GJ68A0, col sep=semicolon] {data/hydrometer-cumulative.csv};
    
    \end{semilogxaxis}
\end{tikzpicture}}
	}

	\caption{Кумулятивные кривые гранулометрического состава исследованных грунтов}
	\label{fig:curves}
	%\raggedright 
	%* ИГЭ-6 --- суглинки тугопластичные, ИГЭ-7 --- суглинки тугопластичные,
	%ИГЭ-8а --- глины полутвердые, ИГЭ-9 --- суглинки полутвердые.
\end{figure}