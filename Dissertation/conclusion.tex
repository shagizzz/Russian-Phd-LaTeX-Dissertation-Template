\chapter*{Заключение}                       % Заголовок
\addcontentsline{toc}{chapter}{Заключение}

%В результате исследований выявлено, что геологическое строение 
%изученного района представляет собой слоистую толщу:
%
%\begin{itemize}
%    \item техногенные (насыпные) грунты (t H);
%    \item верхнечетвертичные покровные отложения (pr III);
%    \item среднечетвертичные флювиогляциальные и озерно-ледниковые отложения московского горизонта (f,lg II ms);
%    \item среднечетвертичные ледниковые отложения московского горизонта (g II ms);
%    \item нижне-среднечетвертичные флювиогляциальные, ледниково-озерные, аллювиальные и озерные отложения донского-московского горизонта (f,lg I-II ds-ms);
%    \item нижнечетвертичные ледниковые отложения донского горизонта (g I ds);
%    \item отложения нижнего отдела меловой системы ($K_1$).
%\end{itemize}
%
%Также было определено, что в районе поселения Сосенское развиты два водоносных горизонта: 
%1) приуроченный к песчаным прослоям в суглинках f,lg II ms; 2) в пылеватых песках $K_1$. 
%Оба горизонта напорные, гидравлическая связь между ними почти повсеместно отсутствует.
%
%Исследуемая территория находится на юго-западе г. Москвы на
%Теплостанской возвышенности, рельеф пологий, сильно расчлененный 
%овражно-балочной сетью.
%
%Участок исследований относится к неопасным по причине отсутствия активных современных 
%геологических процессов.

%Инженерно-геологические условия 
%территории поселения Сосенское относятся 
%ко \RomanNumeralCaps{2} категории сложности по 
%классификации СП 47.13330.2012 "Категории сложности 
%инженерно-геологических условий"{}, так как 
%по геоморфологическим факторам в сфере взаимодействия
%зданий и сооружений с геологической средой  площадка отвечает II категории
%сложности, по геологическим --- II категория сложности (не более 4 
%различных по литологии слоев),
%по гидрогеологическим условиям --- II категория сложности,
%по фактору "геологические и
%инженерно-геологические процессы"{} --- I категория
%сложности (роцессов на территории исследования
%не обнаружено), по фактору "специфические грунты"{} 
%--- I категория сложности (специфических грунтов нет), по фактору «Tехногенные
%воздействия и изменения освоенных территорий» участок отвечает II категории
%сложности (обширные техногенные воздействия на
%территорию исследования).

По совокупности подробно рассмотренных в работе 
инженерно-геологических условий, территория поселения 
Сосенское (г. Москва) относится ко \RomanNumeralCaps{2} категории сложности 
(средняя) по классификации СП 47.13330.2012 "Инженерные 
изыскания для строительства. Основные
положения"{}.

В процессе работы были определены физические свойства
грунтов (влажность грунта, верхний предел пластичности, 
нижний предел пластичности, число пластичности, 
показатель текучести, коэффициент водонасыщения, 
коэффициент пористости, плотность грунта, 
плотность твердых частиц грунта), гранулометрический и 
минеральный составы. Основная работа была проделана по 
определению физико-механических свойств, 
в частности, параметров переуплотнения. Определения 
характеристик переуплотнения проводились 
двумя методами Казагранде и Беккера. 
Напряжения предуплотнения $\sigma_c$ и переуплотнения $POP$ 
изменяются с глубиной, прямо связаны с плотностью и 
обратно --- с влажностью грунта. 
По классификации Г.Г.Болдырева (2014) 
исследованные грунты называются переуплотненными, 
так как их коэффициент переуплотнения $OCR$ (2,4) и больше 1.

Грунты, относящиеся к ледниковым 
отложениям донского горизонта, имеют влажность равную 17\%,
природную плотность --- 2,15 г/\si{\centi\meter^3}, 
плотность твердых частиц --- 2,71 г/\si{\centi\meter^3},
число пластичности и показатель текучести 0,18 и 0,12 соответственно, 
коэффициент пористости --- 0,48, степень водонасыщения --- 0,94.
Значение напряжения предуплотнения --- 686,0 кПа, 
напряжения переуплотнения --- 408,8 кПа,
коэффициент переуплотнения — 2,9.

В итоге, изучая инженерно-геологические особенности территории поселения Сосенское, 
можно сделать вывод, что район благоприятен для строительства жилых комплексов 
с точки зрения инженерно-геологических условий. 