\chapter*{Гидрогеологические условия территории поселения Сосенское} 

На исследуемом участке выделяются следующие водоносные горизонты:

1) Водоносный горизонт f,lgQIIms приурочен к песчаным прослоям в суглинках московского горизонта. 
Горизонт напорный и вскрывается всеми скважинами на глубине 2,7-3,1 (181,7-182,2 \si{\meter}) [2]. 
Пьезометрический уровень устанавливается на глубине 1,3-1,5\si{\meter} (183,2-183,7\si{\meter}) [2]. 
В глинистых грунтах горизонт имеет спорадическое распространение. Относительным водоупором служат ледниковые (моренные) 
и флювиогляциальные суглинки и глины. Питание водоносного горизонта происходит в основном за счет инфильтрации атмосферных осадков. 
Разгрузка осуществляется за пределами участка исследований.

Химический состав воды характеризуется как сульфатно-хлоридно-гидрокарбонатный магниево-кальциевый пресный 
с минерализацией 0,9г/л, рН=7,9 [2]. Вода неагрессивная к бетону на портландцементе любых марок, слабоагрессивная 
к железобетонным конструкциям при периодическом смачивании. Отмечается средняя коррозионная агрессивность 
к свинцовым и высокая к алюминиевым оболочкам кабелей [2].

2) Водоносный горизонт нижнего мела ($K_1$) приурочен к пескам пылеватым. Горизонт напорный и вскрывается 
на глубине 27,8м (157,2\si{\meter}) [2]. Уровень устанавливается на глубине 8,8\si{\meter} (176,2 \si{\meter}) [2]. Высота напора составляет 19,0\si{\meter} [2]. 
По фондовым данным региональным водоупором служат верхнеюрские глины [2].

Химический состав воды характеризуется как гидрокарбонатный магниево-кальциевый пресный с минерализацией 0,6 г/л, 
рН=7,2 [2]. Вода неагрессивная к бетону на портландцементе любых марок, слабоагрессивная к железобетонным конструкциям 
при периодическом смачивании. Отмечается низкая коррозионная агрессивность к свинцовым и средняя к алюминиевым оболочкам кабелей [2].

Гидравлическая связь между четвертичными и меловыми водоносными горизонтами почти повсеместно отсутствует.