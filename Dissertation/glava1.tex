\chapter{Физико-географические условия территории поселения Сосенское}\label{ch:ch1}
 
Участок исследований расположен в юго-заподной части г.~Москвы, НАО, поселение Сосенское, 
вблизи д. Николо-Хованское. Ближайшая станция метро – Филатов луг.

В геоморфологическом отношении исследуемая территория находится в пределах флювиогляциальной равнины. 
Вблизи протекают небольшие постоянные водотоки: р. Сетунь, Карпов ручей.

Изучаемая территория имеет ровную спланированную горизонтальную поверхность с небольшим уклоном на юго-восток, 
с диапазоном абсолютных отметок 184,7-185,0 м [2]. Участок свободен от построек.

Климат Москвы — умеренно-континентальный. 
Характеризуется теплым летом и умеренно-холодной зимой.
По данным многолетних наблюдений (г. Москва) минимальная среднемесячная температура воздуха наблюдается в январе минус 7,8 °С, максимальная — в июле 18,7 °С [2]. 
Количество осадков холодного периода года (с ноября по март) — 225 мм, теплого (с апреля по октябрь) — 465 мм [2]. 

Среднемесячные и среднегодовая температура воздуха в г. Москва (согласно СП 131.13330.2012, таблица 5.1) 
представлены в таблице \ref{tab:temperature}.

%\begin{table} [htbp]% Пример записи таблицы с номером, но без отображаемого наименования
%   \centering
%   \begin{threeparttable}% выравнивание подписи по границам таблицы
%     \captiondelim{}% должен стоять до самого пустого caption
%     \caption{}%
%     \label{tab:temperature}%
%     \begin{SingleSpace}
%       \begin{tabular}{| r | r | r | r | r | r | r | r | r | r | r | r | p{3cm} |}
%       \hline
%       \multicolumn{12}{с}{Среднемесячная температура, \si{\degreeCelsius}} &  \multicolumn{1}{c}{\begin{tabular}[c]{@{}c@{}}Среднегодовая \\ температура,\end{tabular}} \\
%       январь & февраль & март & апрель & май & июнь & июль & август & сентябрь & октябрь & ноябрь & декабрь & \\ \hline
%       -7.8   & -7.1  & -1.3  & 6.4  & 13.0  & 16.9  & 18.7  & 16.8  & 11.1  & 5.2  & -1.1  & -5.6  & 5.4\\ \hline
%       \end{tabular}%
%     \end{SingleSpace}
%   \end{threeparttable}
%\end{table}

 %\begin{table}[]
 % \begin{tabular}{lllllllllllll}
 % \multicolumn{12}{c}{Среднемесячная температура,}                                                       & \multicolumn{1}{c}{\begin{tabular}[c]{@{}c@{}}Среднегодовая \\ температура,\end{tabular}} \\
 % январь & февраль & март & апрель & май  & июнь & июль & август & сентябрь & октябрь & ноябрь & декабрь &                                                                                           \\
 % -7,8   & -7,1    & -1,3 & 6,4    & 13,0 & 16,9 & 18,7 & 16,8   & 11,1     & 5,2     & -1,1   & -5,6    & 5,4                                                                                      
 % \end{tabular}
 %\end{table}

 %\begin{table}[]
 % \captiondelim{}
 % \caption{}
 % \begin{tabular}{ccccccccccccc}
 % \multicolumn{12}{c}{Среднемесячная температура,}                                                       & \begin{tabular}[c]{@{}c@{}}Среднегодовая \\ температура,\end{tabular} \\
 % \begin{sideways} январь \end{sideways} & \begin{sideways} февраль\end{sideways} & \begin{sideways} март\end{sideways} & \begin{sideways} апрель\end{sideways} & \begin{sideways} май\end{sideways}  & \begin{sideways}июнь\end{sideways} & \begin{sideways}июль\end{sideways} & \begin{sideways}август\end{sideways} & \begin{sideways}сентябрь\end{sideways} & \begin{sideways}октябрь\end{sideways} & \begin{sideways}ноябрь\end{sideways} & \begin{sideways}декабрь\end{sideways} &                                                                       \\
 % -7,8   & -7,1    & -1,3 & 6,4    & 13,0 & 16,9 & 18,7 & 16,8   & 11,1     & 5,2     & -1,1   & -5,6    & 5,4                                                                  
 % \end{tabular}
 %\end{table}

\begin{table}[]
  \centering
  \small
  \begin{threeparttable}
  %\captiondelim{}
  \caption{Среднемесячные и среднегодовая температура воздуха в г.Москва (СП~131.13330.2012)}
  \label{tab:temperature}
  \begin{tabular}{|c|c|c|c|c|c|c|c|c|c|c|c|c|}
  \hline
  \multicolumn{12}{|c|}{Среднемесячная температура,\si{\degreeCelsius}}                                                     & \begin{tabular}[c]{@{}c@{}}Среднегодовая \\ температура,\si{\degreeCelsius} \end{tabular} \\ \hline
  \begin{sideways} январь \end{sideways} & \begin{sideways} февраль\end{sideways} & \begin{sideways} март\end{sideways} & \begin{sideways} апрель\end{sideways} & \begin{sideways} май\end{sideways}  & \begin{sideways}июнь\end{sideways} & \begin{sideways}июль\end{sideways} & \begin{sideways}август\end{sideways} & \begin{sideways}сентябрь\end{sideways} & \begin{sideways}октябрь\end{sideways} & \begin{sideways}ноябрь\end{sideways} & \begin{sideways}декабрь\end{sideways} &                                                                       \\ \hline
  -7,8   & -7,1    & -1,3 & 6,4    & 13,0 & 16,9 & 18,7 & 16,8   & 11,1     & 5,2     & -1,1   & -5,6    & 5,4                                                                   \\ \hline
  \end{tabular}
\end{threeparttable}
\end{table}

Вблизи исследуемого участка протекают реки Сетунь и Сосенка. 
Сетунь является правым притоком реки Москвы, расположена на северо-западе от поселения Сосенское. 
Река Сосенка впадает в реку Десну, которая в свою очередь впадает в реку Пахру, а Пахра впадает уже в реку-Москву. 
У реки Сетунь длина – 38 км, площадь бассейна составляет 190 \si{\kilo\meter^2}. 
Длина реки Сосенка – 20 км, площадь бассейна – 107 \si{\kilo\meter^2} [2].