\chapter*{Тектоническое строение и история развития территории поселения Сосенское}

Территория Москвы расположена  в центральной части
Русской плиты, которая включает всю восточную и знчительный фрагмент южной части древней
Восточно-Европейской платформы. 
%Границы плиты хорошо определены: на северо-западе 
%она граничит со сводовым поднятием Балтийского щита, на северо-востоке - возвышенностями 
%Тиманского кряжа, на востоке - горной системой Урала, на юге - молодой Скифской плитой 
%и горными сооружениями Кавказа, Крыма и Карпат.

Тектоническое строение района города Москвы характеризуется наличием двух резко различных
структурных этажей: древнего, докембрийского кристаллического фундамента, погребенного 
на глубину более 1 \si{\kilo\meter}, и залегающего поверх него осадочного чехла.
Оба этажа сложены неоднородными комплексами разновозрастных и разнотипных горных пород, 
расположенных в сложных пространственных соотношениях.

\textbf{Кристаллический фундамент}

Кристаллический фундамент под городом Москва сложен нижне- и среднеархейскими и 
нижнепротерозойскими горными породами, которые первоначально представляли из себя 
осадочные породы, а с течением времени претерпели измениния, связанные с региональным 
метаморфизмом. Распространение и структура метаморфических горных пород изучены в 
основном по особенностям гравитационного и геомагнитного полей, изучение которых свидетельствует 
о сложной дислоцированности пород. Вещественный же состав пород фундамента изучен по 
единичным скважинам [3]. 

Кристаллический фундамент территории Москвы характеризует первый этап развития территории,
который продолжался более 3 миллиардов лет и завершился около 700 миллионов лет назад.
%Этот промежуток времени описывает период активных тектоно-магматических процессов, 
%в результате которых образовалась земная кора континентального типа. 

\textbf{Осадочный чехол}

Чехол начал формироваться в верхнем докембрии. По совокупности 
признаков в районе города Москвы выделяются среднерифейский, верхнерифейско-древлянский 
и валдайский структурные комплексы, образовывшиеся в байкальскую тектоническую эпоху. 
Древлянский и валдайский комплексы относятся к венду.

На временной границе протерозоя и палеозоя в области рассматриваемой части платформы 
произошли глобальные изменения. После заметного 
дифферанциированного прогибания земной коры и накопления первых толщ осадочного чехла
мощностью 350-500 \si{\meter} произошло длительное и медленное поднятие, охватившее
большую часть платформы. В пределах этой поднявшейся области преобладали процессы 
эрозии и денудации, в итоге вызвавшие уничтожение фрагмента протерозойских отложений. 
Только в конце раннего девона из-за изменения поля тектонических движений центральные 
зоны Восточно-Европейской платформы
начинают длительно прогибаться. В результате в течение более 100 миллионов лет 
со среднего девона до позднего карбона здесь доминировал морской режим [3].

На территории Москвы в мезозойскую эру формировались лишь юрские и меловые отложения, это объясняется 
вышеуказанными особенностями развития Московской синеклизы. Также на сохранность 
отложений юры и мела повлияли последующая эрозия и покровные оледенения.
Породы юрской системы с заметным стратиграфическим перерывом и угловым несогласием залегают 
на средне- и верхнекаменноугольных отложениях. Наиболее древние из юрских толщ 
сглаживают неровности рельефа каменоугольных пород. Б.М.Даньшин (1947) в качестве 
основного элемента рельефа территории Москвы в предъюрское время выделил крупную 
субширотную палеодолину. В разрезе мезозойских пород в районе города Москвы 
обособляются три основных литолого-стратиграфических комплекса: алеврито-песчаный 
бат-среднекелловей, глинистый среднекелловей-нижнекимериджский и алеврито-песчаный 
титон-меловой [3].

В кайнозойский период Московский район является частью обширной континентальной равнины,
в пределах которой преобладали процессы денудации, незначительной эрозии и выветривания.
Палеогеновые и неогеновые отложения в окрестностях города Москвы не известны.
Важнейшими событиями четвертичного периода на территории Москвы являлись 
обширные оледенения, которые стали определяющими факторами в формировании 
строения четвертичных отложений и рельефа. В районе города Москвы и ближайшего 
Подмосковья встречаются отложения трех оледенений, на исследуемой территории,
послеление Сосенское, развиты два из них. Возраст отложений этих трех оледенений 
до сих пор исследователями трактуется по-разному. 

Пориод, начившийся с конца докембрия и продолжавшийся весь фанерозой, 
составляет второй этап развития 
территории, который характеризуется платформенным развитием территории и отличается 
медленными слабо контрастными колебаниями земной коры, широким развитием морских
мелководных и континентальных отложений и слабыми деформациями. В позднем докембрии 
и в среднем палеозое земная кора платформы подверглась образованию авлакогенов 
и активизации магматизма. 

В течение всего фонерозоя Восточно-Европейская платформа испытывала 
ощутимое динамическое воздействие на граниицах с активными тектоническими
поясами: в раннем палеозое на северо-западе со стороны Норвежских 
каледонид, в среднем и позднем палеозое на востоке со стороны герцинид 
Урала, в мезозой-кайнозое на юге и юго-западе со стороны Альпийского 
подвижного пояса. За счет этого воздействия изменялись план и рельеф платформы, 
положение и компоновка суши и моря, поднятий и впадин, границы распространения 
разновозрастных отложений и их состав, литологические особенности.