\chapter{Геологическое строение территории поселения Сосенское}\label{ch:ch2}

\section{Стратиграфия отложений}\label{sec:ch2/sec1}

Геологический разрез исследуемой территории был изучен на глубину 29,0 \si{\meter} и представлен следующими стратиграфическими подразделениями:

-техногенные (насыпные) грунты (tQIV);

-верхнечетвертичные покровные отложения (prQIII);

-среднечетвертичные флювиогляциальные и озерно-ледниковые отложения московского горизонта (f,lgQIIms);

-среднечетвертичные ледниковые отложения московского горизонта (gQIIms);

-нижне-среднечетвертичные флювиогляциальные, ледниково-озерные, аллювиальные и озерные отложения донского-московского горизонта (f,lgQI-IIds-ms);

-нижнечетвертичные ледниковые отложения донского горизонта (gQIds);

-отложения нижнего отдела меловой системы ($K_1$).

Поверхность исследуемого участка покрыта почвенно-растительным слоем. Во всех скважинах встречены насыпные грунты (tQIV), представленные суглинком мягкопластичным, перекопанным, с прослоями песка, с включениями обломков бетона, битого кирпича, стекла, остатков древесины. Мощность насыпных грунтов составила 0,2-0,3м [2]. Абс. отметка подошвы насыпных грунтов равна 184,5-184,7м [2].

Верхнечетвертичные покровные отложения (prQIII) залегают повсеместно под насыпными грунтами и представлены глинами полутвердыми.

Глины коричневые с прослоями и пятнами серых глин, полутвердые, местами до тугопластичных, с прослоями ожелезнений. Мощность глин составляет 1,6-1,7\si{\meter} с абс. отм. подошвы 182,8-183,0\si{\meter} [2].

Среднечетвертичные флювиогляциальные и озерно-ледниковые отложения московского горизонта (f,lgQIIms) залегают ниже по разрезу и представлены переслаивающейся толщей супесей, суглинков тугопластичных и песков мелких.

Пески мелкие образуют линзу мощностью 0,2\si{\meter} [2]. Пески мелкие, светло-коричневые, коричневые, средней плотности, глинистые, с прослоями суглинка и супеси, средней степени водонасыщения. Мощность песков составляет 0,2\si{\meter} [2].

Супеси пластичные вскрыты в подошве флювиогляциальных отложений в виде прослоя мощностью 0,6\si{\meter} [2]. Супеси светло-коричневые, коричневые, пластичные, слоистые, песчанистые.

Суглинки тугопластичные слагают основную толщу флювиогляциальных отложений в виде слоя мощностью 0,2-1,7м [2]. Суглинки коричневые, светло-коричневые, тугопластичные, пылеватые, слоистые.

Общая мощность флювиогляциальных отложений московского горизонта составляет 1,3-1,7\si{\meter} с абс. отм. подошвы 181,1-181,7\si{\meter} [2].

Среднечетвертичные ледниковые отложения московского горизонта (gQIIms) залегают под флювиогляциальными отложениями. Они представлены суглинками тугопластичными.

Суглинки от коричневых до красновато-коричневых, тугопластичные, местами полутвердые, песчанистые, с прослоями супеси и песка, с включениями дресвы и щебня преимущественно карбонатных пород до 10-15\%. В подошве с прослоями песка мелкого коричневого насыщенного водой. Их мощность колеблется от 0,9\si{\meter} до 2,1\si{\meter} с абс. отм. подошвы 179,6-180,3\si{\meter} [2].

Нижне-среднечетвертичные флювиогляциальные, ледниково-озерные, аллювиальные и озерные отложения донского-московского горизонта (f,lgQI-IIds-ms) залегают повсеместно под флювиогляциальными и ледниковыми отложениями московского горизонта. Они представлены суглинками тугопластичными в кровле и в подошве, и глинами полутвердыми в средней части.

Суглинки тугопластичные залегают преимущественно в кровле и подошве флювиогляциальных отложений. Суглинки от светло-коричневых до зеленовато-коричнево-серых, тугопластичные, местами до полутвердых, песчанистые, слабо слоистые, с редкими включениями гравия и гальки. Мощность суглинков колеблется от 1,1 до 3,4\si{\meter} [2].

Глины полутвердые преимущественно залегают в средней части толщи флювиогляциальных отложений донского-московского горизонта и вскрыты всеми скважинами. Глины серые, темно-серые, с зеленоватым оттенком, полутвердые, в кровле 0,2м тугопластичные, слоистые, пылеватые, с единичными включениями гравия и гальки. Мощность глин составила 2,8-3,8\si{\meter} [2].

Общая мощность флювиогляциальных отложений донского-московского горизонта составляет 6,7-8,2\si{\meter} с абсолютной отметкой подошвы 171,6-173,4\si{\meter} [2].

Ниже залегают нижнечетвертичные ледниковые отложения донского горизонта (gQIds). Они представлены мощной толщей суглинков полутвердых.

Суглинки коричневые, до темно-серо-коричневых, полутвердые, в кровле 0,1-0,3\si{\meter} тугопластичные, песчанистые, с включениями дресвы и щебня преимущественно карбонатных пород до 15\% [2]. Мощность этих суглинков составила 14,4 \si{\meter}, с абсолютной отметкой подошвы 157,2\si{\meter} [2].

Завершают разрез на изученную глубину (29,0\si{\meter}) отложения нижнего отдела меловой системы ($K_1$) [2]. Они представлены песками пылеватыми. Пески пылеватые, темно-серые, плотные, глинистые, слюдистые, насыщенные водой. Максимально вскрытая мощность этих песков составила 1,2\si{\meter} [2].
