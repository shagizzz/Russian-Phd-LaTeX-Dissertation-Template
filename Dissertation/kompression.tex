\section{Физико-механические свойства}\label{sec:ch6/sec2}

\subsection{Компрессионное сжатие}

Исследование грунта методом компрессионного сжатия осуществляют по ГОСТ 12248-2010.
%следующих характеристик деформируемости в соответствии с
%заданием и программой испытаний: 
%коэффициента сжимаемости \textit{$m_0$}, модулей
%деформации \textit{$E_{oed}$} и \textit{$E_k$} для ветвей первичного и повторного нагружения,
%коэффициентов фильтрационной и вторичной консолидации \textit{$c_\nu$} и \textit{$c_\alpha$} для
%песков мелких и пылеватых, глинистых грунтов, органо-минеральных и
%органических грунтов.

%Эти характеристики вычисляют по результатам исследований
%образцов грунта в компрессионных приборах (одометрах), исключающих
%возможность бокового расширения образца при его нагружении вертикальной
%нагрузкой.
%Результаты испытаний должны быть оформлены в виде графиков
%зависимостей 
%деформаций образца от нагрузки при определении \textit{$m_0$} и \textit{E} и их
%изменения во времени при определении значений \textit{$c_\nu$} и \textit{$c_\alpha$}.

Компрессионное сжатие проводилось в приборах ГТ1.1.4-01 ООО «НПП ГеоТек» (Пенза).
Состав установки для компрессионного испытания должен включать: 
- компрессионный прибор (одометр), состоящий из рабочего кольца с
внутренними размерами подходящего по размерам к образцу (диаметр не менее 70 \si{\milli\meter} 
и отношение диаметра к высоте должно составлять от 2,8 до 3,5), цилиндрической обоймы,
перфорированных вкладыша под рабочее кольцо и штампа (пористых
пластин) и поддона с емкостью для воды; механизм для вертикального нагружения образца грунта; 
устройства для измерения вертикальных деформаций образца грунта.
Конструкция компрессионного прибора должна обеспечивать: подачу воды к образцу снизу и ее отвод; 
герметичность деталей прибора; центрированную передачу нагрузки на штамп; постоянство давления 
на каждой ступени; первоначальную нагрузку на образец от штампа и закрепленных на нем
измерительных приборов не более 0,0025 МПа; перфорация пористых штампов должна 
обеспечивать свободный отток отжимаемой воды из образца.
Перед испытанием компрессионные приборы тарируют на сжатие металлического 
вкладыша, покрытого двумя бумажными фильтрами.

Образцы грунта подготавливают перед испытанием следующим образом:
образец вырезают металлическим кольцом диаметром 70 \si{\milli\meter} 
и высотой 20 \si{\milli\meter}. С обеих сторон образец покрывают 
влажными бумажными фильтрами. 

Нагружение образца осуществлялся степенями нагрузки равномерно. 
Сперва грунт нагружают до напряжения равного природному 
напряжению, рассчитываемому по формуле: 

\[
   \sigma = \rho g h
\]

где $\rho$ --- плотность грунта г/\si{\centi\meter^3},
$g$ --- ускорение свободного падения, \si{\meter}/\si{\second^2},
$h$ --- глубина, с которой был отобран образец, \si{\meter}.

После достижения напрежения в грунте природного напряжения, нагрузка 
проводилась равными ступенями до максимально возможной. На приборе 
компрессионного сжатия ГТ1.1.4-01 ООО «НПП ГеоТек» (Пенза) максимальная 
величина нагрузки равна 2 \si{\mega\pascal}.

После максимального нагружения образца грунта необходимо разгрузить
его. И цикл нагрузка-разгрузка может повториться, это зависит от 
желания исследователя и особенностей грунта.

Результатами испытаний являются разные графики, каждый из которых 
помогает определять ту или иную характеристику. 
График зависимости деформаций образца от нагрузки 
определяет коэффциент сжимаемости $m_0$, 
модули деформации $E$.
Для определения характеристик $m_0$, $E_{oed}$ и $E_k$ 
по результатам
испытания для каждой ступени нагружения вычисляют:
абсолютную вертикальную стабилизированную деформацию образца
грунта $\Delta h$, \si{\milli\meter}, как среднеарифметическое значение показаний
измерительных устройств за вычетом поправки на деформацию
компрессионного прибора $\Delta$; относительную вертикальную деформацию образца грунта \(\epsilon_i = \Delta h/h\);
коэффициент пористости грунта $e_i$ при давлениях $p_i$ по формуле:

\[
   e_i = e_0 - \epsilon_i(1+e_0)
\]

По полученным данным строят график \(\epsilon = f(p)\) или
\(e = f(p)\). Проводят осредняющую плавную кривую.

Коэффициент сжимаемости $m_0$, \si{\mega\pascal^{-1}}, на каждой ступени нагрузки
от $p_i$ до $p_{i+1}$ вычисляют с точностью 0,001 \si{\mega\pascal^{-1}} по формуле:

\[
   m_0 = \frac{e_i - e_{i-1}}{p_{i+1} - p_i}
\]

где $e_i$ и $e_{i+1}$ --- коэффициенты пористости, соответствующие давлениям $p_i$ и $p_{i+1}$.

Одометрический модуль деформации $E_{oed}$ и модуль деформации
по данным компрессионных испытаний $E_k$, \si{\mega\pascal}, в заданном интервале
давлений $\Delta p$ вычисляют с точностью 0,1 \si{\mega\pascal} по формулам:

\[
   E_{oed} = \frac{\Delta p}{\Delta \epsilon}
\]

\[
   E_k = E_{oed} \beta
\]
или
\[
   E_k = \frac{1+e_0}{m_0}\beta
\]
где $\Delta \epsilon$ --- изменение относительного сжатия, соответствующее $\Delta p$;
$m_0$ --- коэффициент сжимаемости, соответствующий $\Delta p$;
$\beta$ --- коэффициент, учитывающий отсутствие поперечного расширения
грунта в компрессионном приборе и вычисляемый по формуле:

\[
   \beta = 1-\frac{2\nu^2}{1-\nu}
\]
где $\nu$- коэффициент поперечной деформации, определяемый по результатам
испытаний в приборах трехосного сжатия или в компрессионных
приборах с измерением бокового давления.
При отсутствии экспериментальных данных $\beta$ допускается принимать
равным 0,8 - для песков; 0,7 - для супесей; 0,6 - для суглинков и 0,4 - для глин.

График зависимости изменения деформаций образца грунта 
от изменения нагрузки определяет значения фильтрационной $c_\nu$ 
и вторичной $c_\alpha$ консолидаций.

Коэффициент фильтрационной консолидации $c_\nu$ и коэффициент
вторичной консолидации $c_\alpha$ определяют в соответствии 
с приложением К ГОСТ 12248-2010.

%График зависимости увеличения работы на единицу объема 
%(произведения давления на деформацию) от приращения вертикального давления
%для определения параметров переуплотнения методом Беккера.
%
%Для этого вычисляется изменение работы на единицу объема для каждого
%приращения деформации по формуле:
%
%\[
%   \Delta A = \frac{\sigma_i + \sigma_f}{2}(\epsilon_f - \epsilon_i)
%\]
%
%$\sigma_i$ — давление, со-
%ответствующее началу приращения де-
%формации (кПа); $\sigma_f$ — давление в кон-
%це приращения деформации (кПа); $\epsilon_i$
%— относительная деформация, соот-
%ветствующая началу приращения; $\epsilon_f$ —
%относительная деформация в конце
%приращения.

%\textbf{Параметры переуплотнения}
\subsection{Параметры переуплотнения}

Для повышения достоверности расчетов грунтовых оснований 
с использованием численного моделирования поведения 
грунтов на базе метода конечных элементов в таких программных
обеспечениях, как PLAXIS, ABAQUS используют следующие 
параметры, характеризующие предварительное напряженное 
состояние грунта:

напряжение предварительного уплотнения $\sigma_c$ 
(максимальное условное эффективное вертикальное напряжение, 
которое грунт испытал в прошлом);

напряжение переуплотнения $POP$;

коэффициент переуплотнения $OCR$.

Величина напряжения переуплотнения $POP$ определяется 
как разница между вертикальным эффективным напряжением 
предуплотнения $\sigma_c$ и вертикальным эффективным бытовым напряжением 
(от собственного веса грунта) на глубине залегания образца $\sigma_0$:

\[
   POP = \sigma_c - \sigma_0
\]

Коэффициент переуплотнения $OCR$ определяется по формуле:

\[
   OCR = \frac{\sigma_c}{\sigma_0}
\]

Первым шагом на пути выявления особенностей формирования грунтов должно быть количественное определение
вертикального напряжения в образце грунта, соответствующее началу перехода от упругих деформаций сжатия к пластическим,
другими словами напряжения предварительного уплотнения $\sigma_c$. 
Существует большое количество методов графической обработки 
результатов компрессионного сжатия для получения $\sigma_c$:

Метод Casagrande (1936) --- наиболее старый метод для расчета структурной 
прочности и давления предварительного уплотнения (предполагается, 
что грунт испытывает изменение прочности, переходя от упругой реакции на нагрузку 
к пластичной, в точке, близкой к напряжению предварительного уплотнения);
Метод Burmister (1951); Метод Schemertmann (1953); Метод Akai (1960);
Метод Janbu (1969); Метод Sеllfors (1975); Метод Tavenas (1979);
Метод Becker (1987) --- метод, определяющий энергию деформации
при каждой нагрузке компрессионных испытаний, применяя зависимость
\(W = f(\sigma')\); и другие методы зарубежных исследователей.
 
В данной работе использовались методы Казагранде и Беккера для определения характеристик 
переуплотнения.

\underline{Метод Казагранде}

Наиболее распространенным способом оценки напряжения предварительного 
уплотнения является метод Казагранде, включенный в ASTM D2435. 
По результатам компрессионного сжатия строится график в полулогарифмическом 
масштабе зависимости коэффициента пористости $e$, д.е. от 
вертикального напряжения $\sigma$, \si{\mega\pascal}. 
На кривой определяется точка, находящаяся на максимальной кривизне 
графика. Через эту точку проводится горизонтальная прямая 
и касательная к компрессионной кривой. 
Через угол $\alpha$ между этими двумя прямыми проводится биссектриса. 
Определяется точка пересечения биссектрисы с продолжением прямолинейного 
участка компрессионной кривой. Проекция данной точки на ось напряжений $\sigma$
и дает величину напряжения предварительного уплотнения $\sigma_c$.

Пашеко-Силва, проанализирующий метод Казагранде, пришел к выводу, 
что получаемое значение напряжения переуплотнения зависит от выбранного
масштаба графика. Также как недостаток можно выделить человеский 
фактор в определении точки максимальной кривизны кривой.


\underline{Метод Беккера}

Беккер предложил метод, по которому значение вертикального эффективного 
напряжения предварительного уплотнения находится по 
графику зависимости увеличения работы на единицу объема 
(произведения напряжения на деформацию) от приращения вертикального напряжения.

Для этого вычисляется изменение работы на единицу объема для каждого
приращения деформации по формуле:

\[
   \Delta W = \frac{\sigma_i + \sigma_f}{2}(\epsilon_f - \epsilon_i)
\]

$\sigma_i$ --- напряжение, соответствующее началу приращения деформации (кПа); 
$\sigma_f$ --- напряжение в конце приращения деформации (кПа);
$\epsilon_i$ --- относительная деформация, соответствующая началу приращения; 
$\epsilon_f$ --- относительная деформация в конце
приращения.

Беккер предположил, что значение напряжения, соответствующего суммарной 
работе, определяется напряжение в конце приращения деформации.
График обычно имеет два прямолинейных участка, пересечение 
двух прямых, проведенных к этим участкам, соответствует 
напряжению предварительного уплотнения. 

Недостатком метода является наличие субъективного фактора при
графических построениях, следовательно, в выборе линейных участков, 
к которым проводятся прямые.