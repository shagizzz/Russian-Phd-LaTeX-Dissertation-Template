\chapter*{Методика лабораторных исследований}

Исследованные грунты относятся к классу дисперсных грунтов, подклассу связных по типу 
к осадочным и к подтипу ледниковых. 
Поэтому опрделения физических и физико-механических свойств проводились по методикам, 
которые используются именно для дисперсных связных грунтов, определенных 
стандартами государства: ГОСТ 5180-2014, ГОСТ 25100-2011, ГОСТ 12536-2014, ГОСТ 12248-2010.

В процессе исследований проводились определения следующих свойств грунтов:

\textbf{1.Влажность грунта.}

Определение влажности производилось по методу высушивания до постоянной массы по ГОСТ 5180-2015. 
Для исследования необходимо: сушильный шкаф, лабораторный весы по ГОСТ 24104, 
металлические или стеклянные бюксы по ГОСТ 25336, шпатель по ГОСТ 10778. 
Перед испытанием необходимо отобрать пробу грунта массой 15-50 г, 
помещают в заранее высушенный, взвешенный (m) и пронумерованный бюкс и плотно закрывают крышкой. 
Если в исследуемом грунте присутствуют включения, 
то при отборе пробы на влажность нужно удалить все видимые включения.

В процессе проведения испытания пробу грунта взвешивают в закрытом бюксе. Затем открытый бюкс 
помещают в нагретый сушильный шкаф. Грунт
высушивают до постоянной массы при температуре (105±2)°С. После каждого высушивания закрытый бюкс 
охлаждают до температуры помещения и взвешивают.
Высушивание проводят до получения разности масс грунта с бюксом при двух последующих взвешиваниях 
не более 0,02 г. Если при последующем взвешивании 
наблюдается увеличение массы, то за результат принимают наименьшую массу.

Влажность грунта w, \% вычисляют по формуле:

\[
   W = \frac{m_1-m_0}{m_0-m}*100
\]
,

где - масса влажного грунта с бюксом, г;
- масса высушенного грунта с бюксом, г;
- масса пустого бюкса, г.
Допускается выражать влажность грунта в долях единицы.
Результаты испытаний следует внести в журнал.

\textbf{2.Верхний предел пластичности.}

Определение верхнего предела плпстичности, то есть влажности грунта на границе текучести 
методом балансирного конуса по ГОСТ 5180-2015.
Границу текучести следует определять как влажность приготовленной из исследуемого грунта пасты, 
при которой балансирный конус погружается
под действием собственной массы за 5 с на глубину 10 мм.

Для исследования грунта необходимо: сушильный шкаф, лабораторный весы по ГОСТ 24104, 
металлические или стеклянные бюксы по ГОСТ 25336, балансирный конус Васильева с цилиндрической 
чашкой, фарфоровая по ГОСТ 9147 или металлическая чашка диаметром 7-8 см, шпатель по ГОСТ 10778,
ступка с пестиком по ГОСТ 9147, сито с отверстием 1 мм.

Образец грунта природной влажности натирают на мелкой терке с добавкой дистиллированной воды 
(вода должна соответствовать ГОСТ 6709 по показателям рН и удельной
электропроводности (УЭП)), если это требуется, удалив из него растительные
остатки крупнее 1 мм, отбирают из размельченного грунта методом
квартования по ГОСТ 8735 пробу массой около 100 г. При наличии в грунтовой
пасте включений размером более 1 мм требуется пропустить грунтовую пасту
сквозь сито с сеткой N 1.

Подготовленную грунтовую пасту тщательно перемешивают шпателем
и небольшими порциями плотно (без воздушных полостей) укладывают в
цилиндрическую чашку. Поверхность пасты заглаживают шпателем вровень с
краями чашки.Балансирный конус, смазанный тонким слоем вазелина, подводят к
поверхности грунтовой пасты так, чтобы его острие касалось пасты. Затем
плавно отпускают конус, позволяя ему погружаться в пасту под действием
собственного веса. Погружение конуса в пасту в течение 5 с на глубину 10 мм показывает,
что грунт имеет влажность, соответствующую границе текучести. По достижении границы текучести 
из пасты отбирают пробы массой 15-30 г для определения влажности.

\textbf{3.Нижний предел пластичности.}

Исследование нижнего предела пластичности, то есть влажности грунта на границе раскатывания 
производится по ГОСТ 5180-2015. Границу раскатываемости определяют как влажность пасты, 
изготовленной из грунта, которая при раскатывании  в жгуты диаметром 3 \si{\milli\meter} 
рападается на отдельные части длиной от 3 до 10 \si{\milli\meter}.

Для определния необходимо следующее оборудование: сушильный шкаф, лабораторный весы по ГОСТ 24104, 
металлические или стеклянные бюксы по ГОСТ 25336, балансирный конус Васильева с цилиндрической 
чашкой, фарфоровая по ГОСТ 9147 или металлическая чашка диаметром 7-8 \si{\centi\meter}, шпатель по ГОСТ 10778,
ступка с пестиком по ГОСТ 9147, сито с отверстием 1 \si{\milli\meter}, мелкая терка и вазелин.

Подготавливают грунт так же, как и при определении верхнего предела пластичности или используют 
подготовленную часть грунта с предыдущего определения массой 40-50 г.
Готовый грунт тщательно перемешивают и раскатывают ладонью по пластмассовой или стеклянной поверхности 
до образования грунта диаметром 3 \si{\milli\meter}. Также возможно раскатывание пальцами одной руки 
на лодони другой. Если жгутики при нужном диаметре не распадаются на части, то грунт собирают 
в комок и повторят раскатывание. После образования кусочков диаметром 3 \si{\milli\meter} и длиной 3-10
\si{\milli\meter} их собирают в бюкс и закрывают крышкой. Когда масса грунта в бюксе достигнет 10-15 г, 
определят влажность этого грунта.

\textbf{4.Плотность грунта.}

Определение проводилось по методу взвешивания грунта в воде по ГОСТ 5180-2015. Для измерения необходимо:
нож, лабораторные весы по ГОСТ 24104, нить, парафин, песчаная баня и штатив. Непосредственно перед 
определением сперва вырезают образец грунта объемом не менее 50 \si{\centi\meter^2} и придают ему сглаженную 
форму без острых углов. Образец обвяззывают нитью так, чтобы оставался свободный конец нити для 
подвязывания нити на штатив. В то же время разогревают парафин до температуры 57-60 \si{\degreeCelsius}.
В процессе определения в первую очередь образец грунта взвешивают, затем покрывают его плотной 
парафиновой оболочкой без пузырьков воздуха. После этого охлажденный образец в оболочке взвешивают. 
Парафинированный образец взвешивают в сосуде с водой.
Для этого над чашей весов устанавливают подставку для сосуда с водой так,
чтобы исключить ее касание к чаше весов. К серьге коромысла
подвешивают образец и опускают в сосуд с водой. Объем сосуда и длина
нити должны обеспечить полное погружение образца в воду. При этом
образец не должен касаться дна и стенок сосуда.
Взвешенный образец вынимают из воды, промокают
фильтровальной бумагой и взвешивают для проверки герметичности
оболочки. Если масса образца увеличилась более чем на 0,02 г по сравнению
с первоначальной, образец следует забраковать и повторить испытание с
другим образцом.

Для расчета плотности грунта  $\rho$,  используют формулу: 

\[
   \rho = \frac{m \rho_p \rho_w}{\rho_p (m_1-m_2)-\rho_w (m_1-m)}\
\]

где m - масса образца грунта до парафинирования, г;
$m_1$ - масса парафинированного образца грунта, г;
$m_2$ - результат взвешивания образца в воде - разность масс
парафинированного образца и вытесненной им воды, г; 
$\rho_p$ - плотность парафина, принимаемая равной 0,900 г/\si{\centi\meter^3};
$\rho_w$ - плотность воды при температуре испытаний,  г/\si{\centi\meter^3}.

%\textbf{5.Гранулометрический состав песчаных грунтов.}

%%Определение гранулометрического состава песчаных грунтов проводитсяситовым методом по ГОСТ 12536-2014.
%%Для исследования необходимо следующее оборудование: сита размером отверстий 10, 5, 2, 1, 0.5, 0,25 и 0,1
%\si{\milli\meter}; весы лабораторные по ГОСТ 241104, весы технические с относительной погрешностью 
%взвешивание не более 0.1 \%; ступка фарфоровая по ГОСТ 9147; пестик по ГОСТ 9147 с резиновым 
%наконечником; чашка фарфоровая по ГОСТ 9147; груша резиновая; кисточка; песчаная баня; 
%шкаф сушильный.
%
%Для подготовки грунта отбирают пробу методом квартавания по ГОСТ 8735. Среднюю массу образца 
%следует определять по таблице 2 из ГОСТ 12536-2014. 
%
%Для выделения частиц песка крупностью от 10 до 0,5 \si{\milli\meter} определяют гранулометрический
%состав без промывки, а крупностью от 10 до 0,1 \si{\milli\meter} - с промывкой водой.
%
%При разделении грунта на фракции без промывки грунт высушивают до воздушно-сухого состояния 
%и растирают пестиком в ступке. Затем отбирают образец методом квартавания и взвешивают на весах. 
%В собранную колонну сит высыпают подготовленный образец песка и просеивают путем легких ударов 
%по колонне. задержавшиеся на ситах частицы снова растирают в ступке и вторично просеивают через сита.
%Полноту просеивания фракций грунта проверяют встряхиванием каждого
%сита над листом бумаги. Если при этом на лист выпадают частицы, то их
%высыпают на следующее сито; просев продолжают до тех пор, пока частицы
%не перестанут выпадать на бумагу.
%Частицы грунта после просеивания взвешивают с каждого сита и с поддона, если масса после анализа 
%превышает первоначальную массу образца более чем на 1 \%, то исследование следует повторить. 
%Потерю грунта при просеивании разносят по всем фракциям пропорционально их массе.
%
%При разделении грунта на фракции с промывкой водой грунт подготавливают, отбирают и взвешивают так же, 
%как без промывки. Подготовленную навеску грунта помещают в ступку и растирают с добавлением 
%воды пестиком с резиновым наконечником. После этого грунт частями помещают на сито с диаметром отверстий
%0,1 \si{\milli\meter} и отмучивают под струей воды до тех пор, пока вода не станет прозрачной.
%Отмученный грунт выпаривают в фарфоровой чашке на песчаной бане. 
%
%Массу частиц грунта размером меньше 0,1 \si{\milli\meter} определяют по разности между массой 
%средней пробы, взятой для анализа, и массой высушенной пробы грунта после промывки. 
%Затем просеивают высушенный грунт через набор сит по той же технологии, что и без промывки водой.
%
%Содержание в грунте каждой фракции рассчитывают по формуле: 
%
%\[
%   A = \frac{g_ф}{g_1}
%\]
%
%где $g_{ф}$ - масса данной фракции грунта, г;
%$g_1$ - масса средней пробы грунта, взятой для анализа, г.
%
%Результаты анализа регистрируют в журнале.

\textbf{5.Гранулометрический состав глинистых грунтов.}

Определение гранулометрического состава песчаных грунтов проводится ареометрическим методом 
по ГОСТ 12536-2014. Гранулометрический (зерновой) состав грунтов
ареометрическим методом проводят путем измерения плотности суспензии
ареометром в процессе ее отстаивания. 

Для исследования необходимо следующее оборудование: ареометр со шкалой 0,995 - 1 - 1,030 и ценой 
деления 0,001; набор сит с поддоном; сита с размером отверстий 10; 5; 2; 1,0; 0,5; 0,25;
0,1 \si{\milli\meter}; весы по ГОСТ 24104; ступка и пест фарфоровые по ГОСТ 9147; пестик 
по ГОСТ 9147 с резиновым 
наконечником; чашка фарфоровая по ГОСТ 9147; эксикатор с силикагель-индикатором по ГОСТ 8984;
шкаф сушильный; колба коническая плоскодонная вместимостью 500 \si{\centi\meter};
воронки диаметром порядка 4 и 14 \si{\centi\meter} по ГОСТ 25336; цилиндр мерный вместимостью 1 л 
и диаметром (60±2)\si{\milli\meter}; термометр с погрешностью до 0,5\si{\degreeCelsius} по ГОСТ 28498;
мешалка для взбалтывания суспензии; секундомер; промывалка; пипетка на 25 мл;
обратный холодильник; 25\%-ный раствор аммиака по ГОСТ 3760;
4\%-ный или 6,7\%-ный пирофосфорнокислый натрий по ГОСТ 342; баня песчаная.


При подготовке к испытанию грунт высушивают до воздушно-сухого состояния 
и растирают комки пестиком в ступке. Затем отбирают образец методом квартавания 
и взвешивают на весах. Навеску грунта просеивают сквозь сита с размером отверстий 10, 5, 
2 и 1 \si{\milli\meter}. После этого частицы грунта, оставшиеся на ситах и в поддоне, взвешивают.
Из грунта, прошедшего через сито 1 \si{\milli\meter} отбирают методом квартавания пробу массой не менее 
30 г. В то же время отбирают грунт на определение гигроскопической или природной влажности по ГОСТ 5180.

Отобранную навеску грунта помещают в колбу емкостью 500 \si{\centi\meter^3} и доливают 200 
\si{\centi\meter^3} дистиллированной воды, добавляют 1 \si{\centi\meter^3} 25\%-го раствора
аммиака. После этого кипятят суспензию в течение 1 часа(для суглинков и глин).

Охлажденную суспензию сливают в стеклянный цилиндр через сито с размером отверстий 
0,1 \si{\milli\meter}. Оставшийся на стенках колбы грунт тщательно вымывают дистиллированной водой 
на сито. Эти частицы тщательно промывают водой, затем помещают в фарфоровую чашку и растирают 
пестиком с резиновым наконечником. Растертые частицы сливают через сито в цилиндр. 
Растирание осадка в чашке и сливание взвеси сквозь сито в цилиндр
следует продолжать до полного осветления воды над частицами,
оставшимися на дне чашки. Объем суспензии в цилиндре не должен превышать 1000 \si{\centi\meter^3}.
Чтобы избежать коагуляцию частиц в суспензию добавляют 5 \si{\centi\meter^3} 
4\%-ный или 6,7\%-ный пирофосфорнокислый натрий по ГОСТ 342. Частицы грунта, не прошедшие через сито, 
смывают в чашку и выпаривают на песчаной бане. Высушенный грунт просеивают через сита 
рамером отверстий 0,5; 0,25 и 0,1 \si{\milli\meter}. Частицы, прошедшие через сито 0,1 \si{\milli\meter} 
переносят в цилиндр.

Суспензию в мерном цилиндре доводят до объема 1000 \si{\centi\meter^3}.


\textbf{6.Плотность твердых частиц грунта.}

Определение плотности частиц грунта производится пикнометрическим методом по 
ГОСТ 12536-2014. Чтобы провести исследование, необходимо следующее оборудование:
пикнометры емкостью 100 или 200 \si{\centi\meter^3} по ГОСТ 22524; сушильный 
шкаф; лабораторные весы по ГОСТ 24104; металлические или стеклянные бюксы по 
ГОСТ 25336; термометр по ГОСТ 28498; песчаная баня; дистиллированная вода 
по ГОСТ 6709; ступка с пестиком по ГОСТ 9147; сито с отверстием 2 \si{\milli\meter} 
по действующей нормативной документации.

При подготовке пробы к определению грунт в воздушно-ухом состоянии растирают в фарфоровой 
ступке, отбирают методом квартавания среднюю пробу массой 100-200 г и просеивают через 
сито с размером отверстий 2 \si{\milli\meter}. Затем отбирают 15 г грунта на каждые 
100 \si{\milli\liter} объема пикнометра и высушивают до постоянной массы. В то же время нужно 
вскипятить дистиллированную воду в течение 1 часа.

Пикнометр наполняют приблизительно на 1/3 часть дистиллированной водой и взвешивают, 
насыпают во внутрь отобранные 15 г грунта и снова взвешивают. Затем взбалтывают пикнометр 
с водой и грунтом и ставят кипятиться на песчаную баню. Пески и суглинки должны кипеть 0,5 ч, 
а суглинки и глины - 1 ч. После кипячения пикнометр охлаждают, доливают дисиллированную воду.
Охлаждать пикнометр до комнатной температуры следует в ванне с водой, температуру воды в 
пикнометре принимают за температуру воды в ванне. Пикнометр закрывают крышечкой, удаляют 
выступившую наружу воду и взвешивают. Далее из пикнометра все выливают, наполняют его 
дистиллированной водой той же температуры. Так же закрывают пикнометр крышкой, удаляют 
лишнюю воду и взвешивают. 

Объем пикнометра \textit{$V_\text{П}$}, \si{\centi\meter^3} рассчитывают по формуле:

\[
   V_\text{П} = \frac{m'_2 - m_\text{П}}{\rho_w}
\]

где $m'_2$ - масса пикнометра с дистиллированной водой - при температуре тарировки, г;
$m_\text{П}$ - масса пустого пикнометра, г;
$\rho_w$ - плотность воды при той же температуре, г/\si{\centi\meter^3}.

Плотность частиц грунта $\rho_s$, г/\si{\centi\meter^3} высчитывают по формуле:

\[
   \rho_s = \frac{\rho_w m_0}{(m_0+m_2-m_1)}
\]

где $m_0$ - масса сухого грунта, г;
$m_1$ - масса пикнометра с водой и грунтом после кипячения при
температуре испытания, г;
$m_2$ - масса пикнометра с водой при той же температуре, г;
$\rho_w$ - плотность воды при той же температуре, г/\si{\centi\meter^3}.

Суспензию в цилиндре взбалтывают мешалкой в течение 1 минуты до полного взмучивания осадка 
со дна цилиндра. Затем по таблице 3 ГОСТ 12536-2014 определяют время взятия замеров по ареометру 
после завершения взбалтывания суспензии. Длительность проведения замера не должна 
превышать 10 с. Также после каждого замера ареометром нужно производить измерение 
температуры суспензии и, если она отличается от 20\si{\degreeCelsius}, то вносят 
поправку в замер по таблице 4 ГОСТ 12536-2014. Помимо этого в отсчеты плотности 
суспензии необходимо внести поправки на нулевое показание ареометра, на высоту 
мениска и диспергатор по приложению Б ГОСТ 12536-2014.

Процентное содержание частиц грунта размером более 10; 10-5; 5-2; 2-1 \si{\milli\meter} 
вычисляют по формуле:

\[
   A = \frac{g_\text{ф}}{g_1}
\]

где $g_\text{ф}$ - масса данной фракции грунта, г;
$g_1$ - масса средней пробы грунта, взятой для анализа, г.

Массу абсолютно сухой средней пробы грунта $g_0$, г вычисляют с
учетом поправки на гигроскопическую влажность при анализе воздушно-сухих
образцов по формуле:

\[
   g_0 = \frac{g_1}{(1+0,01W}
\]

где $g_0$ - масса средней пробы грунта в воздушно-сухом состоянии (или
природной влажности), г;
$W$ - гигроскопическая (или природная) влажность, \%.

Процентное содержание частиц размером более 0,5; 0,25 и 0,1 \si{\milli\meter} 
\textit{X}, \% рассчитывают по формуде: 

\[
   X = \frac{g_\text{п}}{(g_0}(100-K)
\]

где $g_\text{п}$ - масса данной фракции грунта, высушенной до постоянной массы, г;
$(g_0$ - масса абсолютно сухой средней пробы грунта (взятой для ареометра), г;
\textit{K} - суммарное содержание фракции грунта размером более 1,0 \si{\milli\meter}, \%.

По замерам ареометром рассчитывают суммарное содержание всех частиц грунта 
менее данного диаметра \textit{X}, \% по формуле:

\[
   X = \frac{\rho_s R_\text{П}}{(\rho_s-\rho_W)\rho_0}(100-K)
\]

где $R_\text{П}$ - показания ареометра с поправками;
$\rho_s$ - плотность частиц грунта, г/см ;
$\rho_W$ - плотность воды, равная 1 г/см ;
$\rho_0$ - масса абсолютно сухой средней пробы грунта, г;*
\textit{K} - суммарное содержание фракции грунта размером более 1,0 мм, \%.

Процентное содержание частиц грунта от 0,05 до 0,01 \si{\milli\meter} вычисляют по разности
между процентным содержанием фракций менее 0,05 мм и менее 0,01 \si{\milli\meter}.
Аналогично рассчитывают процентное содержание частиц грунта 0,01-0,002
\si{\milli\meter} и 0,002-0,001 \si{\milli\meter}.
 
\textbf{Рассчетные показатели.}

\underline{Число пластичности.}

 Значение числа пластичности \textit{$I_p$}, \% рассчитывается по формуле:

 \[
    \textit{$I_p$} = \textit{$w_L$} - \textit{$w_p$}
 \]

 где \textit{$w_L$} - влажность на границе текучести, \%;
 \textit{$w_p$} - влажность на границе раскатывания, \%.
 
\underline{Показатель текучести.}

 Показатель текучести \textit{$I_L$}, д.е. - показатель консистенции глинистых грунтов.
 Формула для определения этого свойства: 

\[
   \textit{$I_L$} = \frac{\textit{w} - \textit{$w_p$}}{\textit{$I_p$}}
\]

где \textit{w} - естественная влажность грунта, \%;
\textit{$w_p$} - влажность на границе раскатывания, \%;
\textit{$I_p$} - число пластичности, \%.

\underline{Коэффициент пористости.}

Показатель коэффициента пористости  \textit{e}, д.е. определяют по формуле: 

\[
   \textit{e} = \frac{\rho_s - \rho_d}{\rho_d}
\]

где $\rho_s$- плотность частиц грунта, г/\si{\centi\meter^3};
$\rho_d$ - плотность сухого грунта, г/\si{\centi\meter^3}.

\textbf{Рентгено-структурный анализ.}

В качестве образцов использовались неориентированные препараты. 
Растертые предварительно образцы набивались в специальные кюветы без использования прессования 
при постоянном контроле качества поверхности для приготовления максимально разориентированного препарата.
Рентгенодифракционный анализ порошковых препаратов проводился при помощи рентгеновского дифрактометра 
ULTIMA-IV фирмы Rigaku (Япония). Рабочий режим – 40 кВ-40 mA, медное излучение, никелевый фильтр, 
диапазон измерений – 3-65\si{\degree} 2$\Theta$, шаг по углу сканирования 0.02\si{\degree} 2$\Theta$, фиксированная система фокусировочных щелей. 
Для ускорения съемки и повышения качества экспериментальных данных использовался полупроводниковый детектор 
нового поколения -  DTex/Ultra: скорость сканирования – 3\si{\degree}2$\Theta$/минуту 

Диагностика минерального состава проводилась методом сопоставления экспериментального и эталонных 
спектров из базы данных PDF-2 в программном пакете Jade 6.5, компании MDI.

Количественный анализ осуществлялся методом полнопрофильной обработки рентгеновских 
картин от неориентированных препаратов по методу Ритвельда [2, 3] в программном продукте программе BGMN.

Анализ выполнен на кафедре инженерной и экологической геологии геологического факультета МГУ им.М.В.Ломоносова
инж. 1 кат. С.А. Гараниной, вед. инж. С.В. Закусиным, ст.н.с., к.г.-м.н. В.В. Крупской.
