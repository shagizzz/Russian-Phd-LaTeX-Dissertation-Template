\chapter{Заключение}\label{ch:ch8}

В результате исследований выявлено, что геологическое строение 
изученного района представляет собой слоистую толщу:

\begin{itemize}
    \item техногенные (насыпные) грунты (t H);
    \item верхнечетвертичные покровные отложения (pr III);
    \item среднечетвертичные флювиогляциальные и озерно-ледниковые отложения московского горизонта (f,lg II ms);
    \item среднечетвертичные ледниковые отложения московского горизонта (g II ms);
    \item нижне-среднечетвертичные флювиогляциальные, ледниково-озерные, аллювиальные и озерные отложения донского-московского горизонта (f,lg I-II ds-ms);
    \item нижнечетвертичные ледниковые отложения донского горизонта (g I ds);
    \item отложения нижнего отдела меловой системы ($K_1$).
\end{itemize}

Также было определено, что в районе поселения Сосенское развиты два водоносных горизонта: 
1) приуроченный к песчаным прослоям в суглинках f,lg II ms; 2) в пылеватых песках $K_1$. 
Оба горизонта напорные, гидравлическая связь между ними почти повсеместно отсутствует.

Исследуемая территория находится на юго-западе г. Москвы на
Теплостанской возвышенности, рельеф пологий, сильно расчлененный 
овражно-балочной сетью.

Участок исследований относится к неопасным по причине отсутствия активных современных 
геологических процессов.

В процессе работы были определены необходимые физические свойства
грунтов (влажность грунта, верхний предел пластичности, 
нижний предел пластичности, число пластичности, 
показатель текучести, коэффициент водонасыщения, 
коэффициент пористости, плотность грунта, 
плотность твердых частиц грунта), гранулометрический и 
минеральный составы. Основная работа была проделана по 
определению физико-механических свойств, 
в частности, параметров переуплотнения. Определения 
характеристик переуплотнения проводились 
двумя методами --- Казагранде и Беккера. 
Оба метода зарекомендовали себя в инженерной 
геологии, но получение характеристик переуплотнения 
нужно проводить параллельно ими обоими. Оба метода 
напрямую зависят от исполнителя, то есть, проводя 
обработку данных разными исследователями, всегда 
будут получаться расхождения в результатах, это 
объясняется тем, что обработка требует дополнительных 
графических построений, которые могут быть 
выполнены по-разному независимыми исполнителями.

В итоге, изучая инженерно-геологические особенности территории поселения Сосенское, 
можно сделать вывод, что район благоприятен для строительства жилых комплексов 
с точки зрения инженерно-геологических условий, поэтому строительство в этом и близ 
расположенных районах развивается быстро и привлекает новых жителей и застройщиков.